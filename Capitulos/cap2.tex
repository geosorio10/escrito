\chapter{El Entrelazamiento y La Mecánica stadística} \label{cap:2}



La mecánica clásica nos habla de un sistema físico definido que para todos los tiempos está especificado. Este sistema evolucionan de manera determinista dadas las ecuaciones de movimiento. Pero lo que sorprende al tratar con  sistemas termodinámicos es que aunque se hable de un sistema clásico este puede mostrar propiedades que dependan de promedios estadísticos \cite{CallenThermo}. La conexión que hay entre el determinismo y estas probabilidades es una discusión que no se ha podido solucionar dado a que han existido varias soluciones pero todas con sus fallas . Los métodos típicos requieren hablar de promedios de ensamble, promedios temporales, aleatoriedad. Todos estos conceptos siguen siendo muy debatidos y parecería que no solucionará ninguna pregunta seguir con ellos \cite{TodaStat}. Por eso se ha estado buscando nuevos fundamentos para la mecánica estadística.
\\
La mecánica estadística tiene como base el postulado de probabilidades iguales; este viene dado por la ignorancia subjetiva que el observador del sistema tiene. Aunque la mecánica estadística no tiene problemas al comparar sus resultados teóricos con los experimentales, el cimiento filosófico en el que reposa da mucho de que hablar dando así posturas diferentes que desde los inicios de su teoría no han podido ser unificados. En este capítulo se expondrá cómo Popescu et al. en \cite{Popescu2006} muestran una posible luz sobre el problema de unificar las perspectivas de la mecánica estadística. La idea principal que se quiere mostrar es cómo se puede reemplazar el postulado de probabilidades iguales por un principio canónico general basado en el entrelazamiento cuántico, al poner el entrelazamiento cuántico como nuevo cimiento ya no se tiene probabilidades subjetivas sino objetivas dadas por la teoría  cuántica. Este nuevo enfoque permite evitar problemas con la ignorancia subjetiva pero también esquiva el problema de ergodicidad, gracias a esto se puede ver un fundamento más claro y sólido. 
\\
Este capítulo solo se enfocará en mostrar cómo la mayoría de los estados del universo están termalizados sin hablar en ningún momento de la evolución de estos estados. Si en general los estados del universo están termalizados se esperaría que cualquier evolución lleve los estados al equilibrio pero en el siguiente capítulo se darán especificaciones de esta conjetura junto con detalles que hacen ver las complicaciones de formalizar las ideas intuitivas que se tienen sobre el equilibrio.
\\

\subsection{Definiciones}

Antes de seguir sería preciso dar la notación que se usará a lo largo de este capítulo y el siguiente. Cuando se tenga un sistema cuántico grande descrito por un espacio de Hilbert $\mathcal{H}$  este se llamará el universo. Este universo será dividido en  dos subsistemas. El primero se llamará el sistema $S$  y el segundo se le dará el nombre de ambiente $E$, en ocasiones $S$ se le dirá subsistema y $E$ se le podrá llamar baño estos nombres son equivalentes a los anteriores. Esto lleva a descomponer $\mathcal{H}$ como $\mathcal{H}=\mathcal{H}_{S} \otimes \mathcal{H}_{E}$, con dimensiones $d_{S}$ y $d_{E}$ respectivamente, se supone que la dimensión del ambiente es mucho mayor que la del sistema. Para evitar espacios de dimensión infinita se introduce un tope para altas energías y así mantener la dimensión finita. Además se eliminarían términos de interacción del Hamiltoniano que lo lleven a subespacios no permitidos. Aún no se ha especificado nada sobre el subsistema o el ambiente (Excepto las proporciones de sus dimensiones)  esto permite decir que cualquier descomposición del espacio de Hilbert especifica un subsistema y un baño. El subsistema S puede ser cualquier cosa desde una partícula hasta el conjunto de partículas distribuidas por todo el baño.
\\
El estado global puro del universo se escribirá como $\ket{\phi(t)}$ en un tiempo t y su matriz de densidad se escribirá $\rho(t)=\ket{\phi(t)}\bra{\phi(t)}$. El estado del sistema se encontrará al hacer una traza parcial del ambiente sobre el estado del universo $\rho_{S}= \Tr _{B} \rho(t)$; similarmente el estado del ambiente está dado por $\rho_{B}=\Tr _{S} \rho(t)$ \cite{WildeInformation}. El promedio temporal del universo es:
\begin{equation}
\omega= \langle \rho(t) \rangle= \lim_{\tau \to \infty} \frac{1}{\tau} \int_{0}^{\tau} \rho (t)dt
\end{equation}
de manera análoga $\omega_{S}$ y $\omega_{B}$ son el promedio temporal para el sistema y el ambiente respectivamente \cite{TodaStat}. También se tiene el promedio $\langle . \rangle_{\phi}$ que es sobre todos los estados $\ket{\phi} \in \mathcal{H}_{R}$ de acuerdo a la medida estándar (unitariamente invariante). Esta medida se usa para hallar volúmenes de conjuntos de estados.

\section{Idea conceptual}

Este nuevo tratamiento de los fundamentos de la mecánica estadística se tiene un universo. Dando como condición que el ambiente sea los suficientemente grande. Este universo está descrito por un estado cuántico puro (se conoce el estado de manera exacta) que obedece una restricción global. Se plantea que el sistema alcanza el equilibrio térmico por medio de la interacción mutua (termalización) es producto del entrelazamiento del sistema y el ambiente  \cite{Popescu2006}. Lo que se presentará más adelante es una definición más rigurosa que ayudará a dar cotas para la expresión "ambiente suficientemente grande".Esta idea permite formular un principio canónico general: el sistema estará termalizado para casi todos los estados puros del universo. esto es soportado por límites cuantitativos. La restricción que se impone no es una específica, esto generaliza los resultados tradicionales dados en la literatura donde se toma por restricción la energía \cite{KardarStat}.
\\
Ya poniendo lo dicho en un contexto un poco más matemático y siguiendo a Popescu et al. se supone tener un universo asilado y bastante grande este tiene dos partes el sistema $S$ y el ambiente $E$. La dimensión del ambiente es mucho más grande que la del sistema. Además se le impone una restricción global al universo llamada $R$. Desde la mecánica cuántica esto puede ponerse como restricciones en el espacio de Hilbert, restricción de los estados posibles:

\begin{equation}
\mathcal{H}_{R}\subseteq \mathcal{H}_{S}\otimes \mathcal{H}_{E},
\end{equation}

Donde $\mathcal{H}_{S}$ y $\mathcal{H}_{E}$ son los espacios de Hilbert del sistema y el ambiente con dimension $d_{S}$  y $d_{E}$ respectivamente. Es bueno recalcar que $R$ es una restricción arbitraria generalmente se toma como la energía del universo. Ahora se define el estado equiprobable del universo bajo $R$ como:

\begin{equation}
\mathcal{E}_{R} = \frac{1}{d_{R}} \mathbbm{1}_{R},
\end{equation}

Donde $\mathbbm{1}_{R}$ es el operador identidad (proyección) sobre el espacio de Hilbert  $\mathcal{H}_{R}$  que tiene dimensión $d_{R}$. Esto se relaciona con el principio de probabilidades iguales porque este es el estado máximamente mezclado en $\mathcal{H}_{R}$  \cite{SakuraiQuantum} por ser así todos los estados bajo la restricción de $R$ tienen la misma probabilidad de salir.
\\
Definimos $\Omega_{S}$ como el estado canónico que está restringido por $R$ cuando el universo se encuentra en el estado $\mathcal{E}_{R}$. Esto significa que si se hace una traza parcial del ambiente al universo da como resultado el estado canónico:

\begin{equation}
 \Omega_{S} =\Tr_{E} \mathcal{E}_{R}.
\end{equation}
\\
Para lo que sigue se hace un supuesto importante  y es que el universo está en un estado puro $\ket{\phi}$ y no en un estado mixto $\mathcal{E}_{R}$, esto quiere decir que se conoce todo lo que es permitido por la mecánica cuántica del universo. Si estuviese en un estado mixto significaría que nosotros no tenemos toda la información que se pudiese tener \cite{SakuraiQuantum}. Ahora lo que se quiere ver es que  pese a que el estado del universo es puro el estado reducido del sistema

\begin{equation}
\rho_{S}=\Tr_{E}\ket{\phi}\bra{\phi},
\end{equation}

se acerca al estado canónico para la gran mayoría de los casos es decir:

\begin{equation}
\rho_{S} \approx \Omega_{S}.
\end{equation}
\\
Por consiguiente para casi todos los estados puros del universo $\ket{\phi} \in \mathcal{H}_{R}$ el sistema se comporta como si el universo estuviese en el estado mixto equiprobable $\mathcal{E}_{R}$. Este  es el principio general canónico. Clarificando lo esbozado, el estado canónico del sistema $\Omega_{S}$ es el estado del sistema cuando el universo se encuentra en el estado equiprobable $\mathcal{E}_{R}$. Se puede interpretar el principio general canónico como un principio que estipula que las probabilidades iguales del sistema son aparentes porque para casi cualquier estado del universo, que sea puro, un subsistema de este universo que cumpla con ser lo suficientemente pequeño se comporta como si el universo estuviese en el estado equiprobable $\mathcal{E}_{R}$. Cabe recordar que aún no se ha especificado la restricción $R$ entonces todo este análisis es general, la restricción no necesariamente debe ser la energía u otras cantidades que se conserven. Esto hace que $\Omega_{S}$ no  deba ser obligatoriamente el estado canónico usual, puede ser el gran canónico o cualquier otro que sea acorde con la restricción impuesta\cite{ReichlStat}. Este principio puede ser de utilidad cuando la interacción entre el ambiente y el sistema no es débil o cuando las interacciones son complicadas el principio general canónico también aplica.

\section{Formulación matemática}

Hasta ahora no se han entrado en los detalles ni en qué significa bastante pequeño o bastante grandes, ni tampoco se ha demostrado el principio general canónico. En esta sección se orientará en los detalles matemáticos, las herramientas usadas y la demostración de los teoremas que Popescu et. al siguienron. En la siguiente sección se dará una perspectiva más física a lo hecho aquí.
\\
Para empezar se debe decir cuál será la distancia que usaremos para darle un sentido de cercanía a los estados $\rho_{S}$ y $\Omega_{S}$. La distancia a usar es una bastante conocida en la teoría cuántica de la información, la distancia de traza \cite{NielsenInformation}. Esta se define como:
\begin{equation}
D(\rho_{S}, \Omega_{S})= \frac{1}{2} \Tr |\rho_{S} -\Omega_{S}|=\frac{1}{2} \Tr \sqrt{(\rho_{S} -\Omega_{S})^{\dag}(\rho_{S} -\Omega_{S})}.
\end{equation}

Esta distancia de traza se relaciona con la distancia de norma de manera sencilla

\begin{equation}
\norm{\rho_{S}-\Omega_{S}}_{1}=2 D(\rho_{S}, \Omega_{S}).
\end{equation}

Teniendo ya una forma de darle sentido al concepto de que dos estados son cercanos entonces se puede plantear el teorema central de \cite{Popescu2006}:
\\
\begin{theorem} \label{teorema principal}
Para un estado escogido de manera aleatoria $\ket{\phi} \in \mathcal{H}_{R} \subseteq \mathcal{H}_{S} \otimes \mathcal{H}_{E} $ y un $\epsilon > 0$ arbitrario, la distancia entre la matriz densidad reducida del sistema $\rho_{S}=\Tr_{E}(\ket{\phi} \bra{\phi})$  y el estado canónico $\Omega_{S}=\Tr_{E} ( \mathcal{E}_{R})$ esta dado probabilísticamente por 

\begin{equation}
Pr_{\phi} \{  \norm{\rho_{S} -\Omega_{S}}_{1} \geq \eta \} \leq \eta',
\end{equation}

Donde 

\begin{equation}
\eta= \epsilon + \sqrt{ \frac{d_{S}}{d_{E}^{eff}} },
\end{equation}

\begin{equation}
\eta'=2\exp (-C d_{R} \epsilon^{2} ).
\end{equation}


y las constantes son: $ C=(18 \pi^{3})^{-1}, d_{R} = \dim \mathcal{H}_{R}, d_{S} = \dim \mathcal{H}_{S} $. $d_{E}^{eff}$ es la medida efectiva del tamaño del ambiente,
\begin{equation}
d_{E}^{eff}= \frac{1}{\Tr \Omega_{E}^{2}} \ge \frac{d_{R}}{d_{S}}.
\end{equation}

Donde $\Omega_{E}= \Tr_{S} \mathcal{E}_{R}$. Ambas cantidades $\eta $ y $\eta'$ serán pequeñas esto implica que el estado estará cercano al estado canónico con alta probabilidad cuando la dimensión efectiva del ambiente sea mucho más grande que la del sistema ( es decir $d_{E}^{eff} >> d_{S}$) y  $d_{R}\epsilon^2>>1>>\epsilon$. Esta última condición se puede asegurar cuando el espacio accesible total es grande ( es decir $d_{R}>>1$), escogiendo $\epsilon=d_{R}^{-\frac{1}{3}}$.\\
\end{theorem}

\subsection{Lema de Levy} \label{ levy}
Para poder demostrar el teorema \ref{teorema principal} se usará  el lema de Levy \cite{Lema}. Este dice que al seleccionar un punto $\phi$ aleatoriamente de una hiperesfera de dimensión alta y que $f(\phi)$ no cambie muy rápido, entonces $f(\phi) \approx \langle f \rangle $ con alta probabilidad, Más exactamente:\\

\begin{lemma} \label{lemma de levy}

Dada una función $f: \mathbb{S}^d \to \mathbb{R} $ definida en la hiperesfera d-dimensional $\mathbb{S}^d$ , y un punto $\phi \in \mathbb{S}^d $ es escogido de manera uniformemente aleatoria,
\begin{equation}
Pr_{\phi} \{ |f(\phi)- \langle f \rangle| \geq \epsilon \} \leq 2 \exp(-\frac{2C(d+1)\epsilon^2}{\eta^2})
\end{equation}
donde $\eta$ es la constante de Lipschitz de $f$, dado por $\eta= \sup|\nabla f|$ y $C=(18 \pi^3)^{-1} $.\\

\end{lemma}

Los conceptos manejados por el teorema \ref{lemma de levy} son conocidos generalmente a excepción de la llamada constante de Lipschitz.Para poder entender qué es la  constante de Lipschitz se debe ver primero qué significa que una función sea Lipschitz continua. 
\\
La definición de continuidad dad en el cálculo básico es:

\theoremstyle{definition}
\begin{definition}{continuidad}
Sea $f: I \to \mathbb{R}$ donde $I$ puede ser un intervalo abierto $(a,b)$ o uno cerrado $[a,b]$, además $C \in I$. Se dice que $f$ es continua en $C$ si y solo si para todo $ \epsilon >0 $  existe un $ \delta >0 $ tal que  $ |x-c|\longrightarrow |f(x)-f(c)|< \epsilon $.
\end{definition} 

La  definición anterior es la usada por lo general pero hay sutilezas en este concepto que no siempre son mostradas; como por ejemplo que $\delta$ depende de donde se ponga el  punto $C$, esto se ve claramente en la siguiente función: $f: (0,1) \to \mathbb{R}$, $f(x)=\frac{1}{x} $ al $C$ estar más lejos del $0$ permite un $\delta$ más grande pero al acercarse al $0$ el $\delta$ debe ser más pequeño. La continuidad de Lipschitz permite que se defina un $\delta$ constante sin importar donde se encuentre el $C$. Para resolver este detalle se motiva la definicón de Lipschitz continuo. Se es Lipschitz continuo con constante $\eta$ si

\begin{equation}
|f(x) -f(y)| \leq \eta |x-y|,
\end{equation}

esto permite decir que  $|f(x)-f(y)| < \epsilon $ entonces $\delta < \frac{\epsilon}{\eta} $. Ahora si f es derivable y $\nabla f$ es acotado, para $x$ y $y$ dados existe $\xi$ entre ellos tal que:


\begin{align}
&\implies f(x)-f(y) = \nabla f(\xi) (x-y) \\
&\implies |f(x)- f(y)| \le |\nabla f(\xi)| |x-y| \\
&\implies |f(x)- f(y)| 	\le \sup|\nabla f(\xi)| |x-y|,
\end{align}

como $\nabla f$ es acotado se tiene que $\sup |\nabla f(\xi)|=\eta$.\\

Gracias a la normalización, los estados puros en $\mathcal{H}_{R}$ se pueden representar como puntos sobre la superficie de una hiperesfera de dimensión $2d_{R}-1$, o sea  $\mathbb{S}^{2d_{R}-1}$ \cite{SakuraiQuantum}. Luego se puede aplicar \ref{lemma de levy} a estado cuánticos $\phi$ aleatoriamente seleccionados. Para los estados seleccionados aleatoriamente $\phi \in \mathcal{H}_{R}$, se desea mostrar que $\norm{\rho_{S}- \Omega_{S}}_{1} \approx 0$ con alta probabilidad. Para poder usar \ref{lemma de levy} primero  debe encontrarse la constante de Lipschitz ya teniendo una idea de qué es ser Lipschitz continuo se encontrará una cota para la constante $\eta$ de la función $f(\phi)=\norm{\rho_{S}-\Omega_{S}}_{1}$ que es la función que nos interesa para el problema físico. Entonces para poder lograr esto se procederá de la siguiente forma se definen dos estados reducidos $\rho_{1}= \Tr_{E} (\ket{\phi_{1}} \bra{\phi_{1}})$ y $\rho_{2}= \Tr_{E}(\ket{\phi_{2}} \bra{\phi_{2}})$, entonces

\begin{equation}
|f(\phi_{1})-f(\phi_{2})|^2= |\norm{\rho_{1}-\Omega}_{1} - \norm{\rho_{2}-\Omega}_{1}|^2.
\end{equation}

como $\norm{M}_{1}$ es una distancia (esto es $d(\rho_{1},\Omega)= \norm{\rho_{1}-\Omega}_{1}$) es cierto para un espacio métrico que

\begin{equation}
|d(x,z)-d(y,z)| \le d(x,y),
\end{equation}

Por lo tanto 

\begin{equation}
\Big | \norm{\rho_{1}-\Omega}_{1} - \norm{\rho_{2}-\Omega}_{1} \Big |^2 \le \norm{\rho_{1}- \rho_{2} }_{1}^2= \norm{\Tr_{E}(\ket{\phi_{1}} \bra{\phi_{1}}-\ket{\phi_{2}} \bra{\phi_{2}})}_{1}^{2}.
\end{equation}

Como existe una cota a la norma de una traza parcial dada por

\begin{equation}
\norm{\Tr_{\mathcal{B}}(M)}_{p} \le [dim(\mathcal{H_{B}})]^{\frac{p-1}{p}} \norm{M}_{p},
\end{equation}

entones la cota sobre los estados reducidos queda 
\begin{equation}
\norm{\Tr_{E}(\ket{\phi_{1}} \bra{\phi_{1}}-\ket{\phi_{2}} \bra{\phi_{2}})}_{1} \le \norm{\ket{\phi_{1}} \bra{\phi_{1}}-\ket{\phi_{2}} \bra{\phi_{2}}}_{1}.
\end{equation}

Por lo tanto se tiene hasta ahora:
\begin{equation}
\norm{\rho_{1}- \rho_{2} }_{1}^2 \le \norm{\ket{\phi_{1}} \bra{\phi_{1}}-\ket{\phi_{2}} \bra{\phi_{2}}}_{1}^{2},
\end{equation}

Usando la hermiticidad de $\rho$ y el teorema espectral se puede descomponer $\rho= UDU^{\dag}$ donde $U$ es un operador unitario y $D$ es diagonal. Junto con las propiedades de la traza $\Tr(\sqrt{UD^2U^{\dag}})= Tr(U\sqrt{D^2}U^{\dag})= \Tr(\sqrt{D^{2}})$ se llega a
\begin{equation}
\norm{\ket{\phi_{1}} \bra{\phi_{1}}-\ket{\phi_{2}} \bra{\phi_{2}}}_{1}^{2} = 4(1-|\bra{\phi_{1}} \ket{\phi_{2}}|^{2})
\end{equation}
entonces
\begin{equation}
 4(1-|\bra{\phi_{1}} \ket{\phi_{2}}|^{2}) \le 4|\ket{\phi}-\ket{\phi}|^2.
\end{equation}

Uniendo todos los pasos anteriores
\begin{equation}
|f(\phi_{1})-f(\phi_{2})|^{2} \le  4|\ket{\phi}-\ket{\phi}|^2
\end{equation}

o sea 
\begin{equation}
|f(\phi_{1})-f(\phi_{2})| \le  2|\ket{\phi}-\ket{\phi}|,
\end{equation}

Con esto se muestra que $\eta \le 2$.\\

\subsection{Demonstración del principio general canónico}
En esta parte se dará una demostración matemática explícita del principio general canónico usando el lema de Levy se entrará en los detalles matemáticos de usar este lema y las cotas adicionales que se necesitan para poder llegar al teorema \ref{teorema principal}. Habiendo dado una cota para $\eta$ ahora se puede usar por completo el lema \ref{lemma de levy} para la función $f(\phi)=\norm{\rho_{S} -\Omega_{S}}_{1}$ recordando que se reemplazará $d$ en el lema por $d=2d_{R}-1$.Tomando la parte derecha de la desigualdad de \ref{lemma de levy}  y sustituyendo la dimensión se tiene:

\begin{equation}
 2 \exp(-\frac{2C(d+1)\epsilon^2}{\eta^2})= 2 \exp(-\frac{4Cd_{R}\epsilon^2}{\eta^2})
\end{equation}

como $\eta \le 2$ entonces

\begin{equation} \label{desigualdad}
2\exp(-Cd_{R}\epsilon^2) \ge 2 \exp(-\frac{4Cd_{R}\epsilon^2}{\eta^2}) \ge \Pr_{\phi}[|f(\phi)- \langle f \rangle| \ge \epsilon ].
\end{equation}

Mirando más atentamente  $|f(\phi)- \langle f \rangle| \ge \epsilon $, como la norma de traza es un distancia se tiene que  $\norm{\rho_{S} -\Omega_{S}}_{1} \ge 0$ entonces


\begin{equation}
\norm{\rho_{S} -\Omega_{S}}_{1}- \langle \norm{\rho_{S} -\Omega_{S}}_{1} \rangle_{\phi} \ge \epsilon.
\end{equation}

Nombrando a $\mu = \epsilon + \langle \norm{\rho_{S} -\Omega_{S}}_{1} \rangle_{\phi} $ y $\mu'=2 \exp(-Cd_{R}\epsilon^2)$ esto permite organizar el lema de levy así:

\begin{equation} \label{eq.3}
\Pr_{\phi}[\norm{\rho_{S} -\Omega_{S}}_{1} \ge \mu] \le \mu'.
\end{equation}

Debido a que $d_{R}>>1$ Se asegura que $\epsilon$ y $\mu'$ son cantidades pequeñas al escoger $\epsilon=d_{R}^{-1/3}$. Para llegar a \ref{teorema principal} falta acotar $\mu$ con las dimensiones de los espacios conocidos, se impondrá una cota a $\langle \norm{\rho_{S} -\Omega_{S}}_{1} \rangle_{\phi}$; lo primero  para lograr esta empresa es acotar este promedio con trazas del estado del sistema  y luego se calcularán estas trazas para poder dejar la cota en términos de las dimensiones del sistema y la dimensión efectiva del ambiente. Se procede a encontrar la relación entre $\norm{\rho_{S} -\Omega_{S}}_{1}$ y $\norm{\rho_{S} -\Omega_{S}}_{2}$ esto se hará para tener una facilidad de manejo ya que la norma de Hilbert-Schmidt ($\norm{.}_{2}$) es más sencilla para trabajar que la norma de traza y luego se procederá con los planeado.
\\
La relación entre estas dos normas se puede ver desde el manejo de matrices. Sea $M$ una matriz $n \times n$ se sabe que si $M$ tiene $\lambda_{i}$ valores propios entonces:
\begin{equation}
\Tr M= \sum_{i} \lambda_{i}
\end{equation}
con esto se puede escribir de manera explícita la norma de traza 
\begin{equation}
\norm{M}_{1}^2=(\Tr|M|)^{2}=n^{2} \Big( \frac{1}{n}\sum_{i} |\lambda_{i}| \Big)^{2}.
\end{equation}
Como la función $x^{2}$ es convexa se puede usar la desigualdad de Jensen que dice: sean $a_{1},a_{2},...,a_{n} \le 0$ constantes y $a_{1} +...+a_{n}=1$ sea $f: I \to \mathbb{R}$ donde I es un intervalo, $x_{1},...,x_{n} \in I$ entonces
\begin{equation}
f(a_{1}x_{1}+...+a_{n}x_{n}) \le a_{1}f(x_{1})+...+a_{n}f(x_{n}),
\end{equation}

con esto se puede decir que

\begin{equation}
\Big( \frac{1}{n}\sum_{i} |\lambda_{i}| \Big)^{2} \le \frac{1}{n}\sum_{i} |\lambda_{i}|^{2}.
\end{equation}
Pero se sabe que 
\begin{equation}
\sum_{i} |\lambda_{i}|^2= \Tr(|M|^2)=\norm{M}_{2}^{2}
\end{equation}
se llega entonces a la conclusión que 
\begin{equation}
\norm{M}_{1}^{2} = n^{2}\Bigg( \frac{1}{n} \sum_{i}| \lambda_{i} | \Bigg)^{2} \le  n^{2} \frac{1}{n} \sum_{i} | \lambda_{i} |^{2}= n\norm{M}_{2}^{2}
\end{equation}
Gracias a lo anterior la relación entre normas es:
\begin{equation}\label{eq:0}
\norm{\rho_{S} - \Omega_{S}}_{1} \le \sqrt{d_{S}}\norm{\rho_{S} - \Omega_{S}}_{2}
\end{equation}
Esta relación se usará un poco más adelante.\\

Volviendo al cálculo de $\langle \norm{\rho_{S} -\Omega_{S}}_{1} \rangle_{\phi}$ se acotará la norma de Hilbert-Schmidt y con la relación entre normas se dará la desigualdad que limite el promedio de la norma de traza. Para empezar se recuerda que $ \langle f^{2} \rangle-\langle f \rangle^{2} \ge 0$ donde $\langle f \rangle = \int_{\mathcal{M}} f(x)p(x)dx$. Tomando a $f$ como $f= \norm{\rho_{S}-\Omega_{S}}_{2}$ entones

\begin{equation}
\langle \norm{ \rho_{S}-\Omega_{S} }_{2} \rangle \le \sqrt{ \langle \norm{ \rho_{S}-\Omega_{S} }_{2}^{2}\rangle}
\end{equation}

acordándose que este promedio es tomado a los estados $\ket{\phi}$, se omitirá por ahora el subíndice indicando este promedio, esto hace que $\Omega_{S}$ se tome constante. Por hermiticidad de $\rho_{S}-\Omega_{S}$ 

\begin{equation}
\sqrt{\langle \norm{ \rho_{S}-\Omega_{S} }_{2}^{2}\rangle} = \sqrt{\langle \Tr(\rho_{S}-\Omega_{S})^{2} \rangle}
\end{equation}

\begin{equation}
\sqrt{\langle \Tr( \rho_{S}-\Omega_{S})^{2} \rangle}= \sqrt{\langle \Tr(\rho_{S})^{2}\rangle -2 \Tr(\langle \rho_{S} \rangle \Omega_{S})+ \Tr(\Omega_{S}^{2}) }
\end{equation}

porque $\langle \rho_{S} \rangle=\Omega_{S}$. Luego se llega a

\begin{equation}
\langle \norm{ \rho_{S}-\Omega_{S} }_{2}\rangle \le \sqrt{\langle \norm{ \rho_{S}-\Omega_{S} }_{2}^{2}\rangle}= \sqrt{\langle \Tr(\rho_{S})^{2}\rangle - \Tr(\Omega_{S}^{2}) }.
\end{equation}

Por la relación entre la norma de traza y la norma de Hilbert-Schmidt se concluye lo que se quería

\begin{equation}
\langle \norm{ \rho_{S}-\Omega_{S} }_{1}\rangle \le \sqrt{d_{S}( \langle \Tr(\rho_{S})^{2}\rangle -\Tr(\Omega_{S}^{2}) ) }.
\end{equation}
\\
Aunque ya se ha acotado $\langle \norm{ \rho_{S}-\Omega_{S} }_{1}\rangle$ se quiere relacionar esta cota con las dimensiones del sistema para esto se procederá a demostrar la desigualdad
\begin{equation}
\langle \Tr \rho_{S}^{2} \rangle \le \Tr \langle \rho_{S} \rangle^{2} +\Tr \langle \rho_E \rangle^{2},
\end{equation}
recordando que el promedio es tomado con respecto a los estados $\ket{\phi}$, los métodos usados para encontrar esta desigualdad son usados también en destilación de entrelazamiento aleatoria y codificación de canal cuántico aleatorio \cite{QuantumDistilation}. Para poder hacer este cálculo se introduce una segunda copia del espacio de Hilbert. Ahora el problema se trabaja en $\mathcal{H}_{R} \otimes \mathcal{H}_{R'}$, donde $\mathcal{H}_{R'} \subseteq \mathcal{H}_{S'} \otimes \mathcal{H}_{E'}$.
Percatándose de lo siguiente

\begin{equation}
\Tr_{S} (\rho_{S})^2 = \sum_{k} (\rho_{kk})^{2} = \sum_{k,l,k',l'}(\rho_{kl})(\rho_{k'l'}) \bra{kk'}\ket{ll'}\bra{l'l'}\ket{kk'}.
\end{equation}

Sea $F_{SS'}$ la operación "flip" $S \longleftrightarrow S'$ definida de esta manera:

\begin{equation}
F_{SS'} =\sum_{S,S'} \ket{s'}\bra{s}_{S} \otimes \ket{s}\bra{s'}_{S'}
\end{equation}

Entonces

\begin{align*} 
\sum_{k,l,k',l'}(\rho_{kl})(\rho_{k'l'}) \bra{kk'}\ket{ll'}\bra{l'l'}\ket{kk'} &= \Tr_{SS'} ((\rho_{S} \otimes \rho_{S'})F_{SS'})\\
\quad &=\Tr_{RR'}((\ket{\phi} \bra{\phi} \otimes \ket{\phi} \bra{\phi})_{RR'}(F_{SS'} \otimes \mathbbm{1}_{EE'}))
\end{align*}



Pero como se quiere $\langle \Tr (\rho)^{2} \rangle= \int \Tr (\rho)^{2} d\phi $. Entonces para resolver esto se requiere saber $V=\int (\ket{\phi} \bra{\phi} \otimes \ket{\phi} \bra{\phi})d\phi$. V puede representarse como:

\begin{equation}
V= \alpha \Pi_{RR'}^{sim} + \beta \Pi_{RR'}^{anti},
\end{equation}
$\alpha$ y $\beta$ son constantes y $\Pi_{RR'}^{sim/anti}$ son proyectores en el subespacio simétrico y antisimétrico de $\mathcal{H_{R} \otimes H_{R'}}$ respectivamente, esto es posible por la invarianza unitaria de V. Debido a que 	
\begin{equation}
(\ket{\phi} \bra{\phi} \otimes \ket{\phi} \bra{\phi})\frac{1}{\sqrt{2}} ( \ket{ab} -\ket{ba})=0  \quad  \forall a,b,\phi
\end{equation}
la parte antisimétrica siempre debe ser 0 entonces $\beta=0$. Por la normalización de $V$ se llega a $\alpha= \frac{1}{dim(RR'_{sim})}$, la dimensión del espacio $RR'_{sim}$ está dada por el álgebra lineal $dim(RR'_{sim})= \frac{d_{R}(d_{R}+1)}{2}$.Entonces
\begin{equation}
V= \langle \ket{\phi} \bra{\phi} \otimes \ket{\phi} \bra{\phi} \rangle = \frac{2}{d_{R}(d_{R} +1)} \Pi_{RR'}^{sim}.
\end{equation}
luego se tiene que 
\begin{equation}
\langle \Tr (\rho)^{2} \rangle = \Tr_{RR'} \Big( \Big( \frac{2}{d_{R}(d_{R} +1)} \Pi_{RR'}^{sim} \Big)(F_{SS'} \otimes \mathbbm{1}_{EE'}) \Big)
\end{equation}
Al ser $\Pi_{RR'}^{sim}$ un proyector simétrico se escribe
\begin{equation}
\Pi_{RR'}^{sim}=\frac{1}{2}(\mathbbm{1}_{RR'}+(F_{RR'} ))
\end{equation}
donde $F_{RR'}$ es el operador "flip"  $R \longleftrightarrow R'$. Como $F_{RR'}$ es un operador que actúa sobre $RR'$ puede escribirse como $F_{RR'}= \mathbbm{1}_{RR'}(F_{SS'} \otimes F_{EE'})$.

reuniendo todo lo anterior

\begin{equation}
\langle \Tr_{S} \rho_{S}^2 \rangle = \Tr_{RR'} \Bigg( \frac{1}{d_{R}(d_{R}+1)} \Big( \mathbbm{1}_{RR'}+\mathbbm{1}_{RR'}(F_{SS'} \otimes F_{EE'}) \Big)  (F_{SS'}\otimes \mathbbm{1}_{EE'})   \Bigg)
\end{equation}

Distribuyendo y sabiendo que al hacer dos veces la operación "flip" no afecta nada se sigue lo siguiente

\begin{equation}
\Tr_{RR'} \Bigg( \frac{\mathbbm{1}_{RR'}}{d_{R}(d_{R}+1)} F_{SS'} \otimes \mathbbm{1}_{EE'} + \frac{\mathbbm{1}_{RR'}}{d_{R}(d_{R}+1)} (\mathbbm{1}_{SS'} \otimes F_{EE'})  \Bigg).
\end{equation}

Por las propiedades aditivas de la traza junto con $\mathbbm{1}_{R} \otimes \mathbbm{1}_{R'}$  y $\frac{1}{d_{R}(d_{R}+1)} \le \frac{1}{d_{R}^{2}}$ se llega a:

\begin{align*}
\langle \Tr_{S} \rho_{S}^2 \rangle &= \Tr_{RR'} \Bigg( \Bigg( \frac{\mathbbm{1}_{RR'}}{d_{R}(d_{R}+1)} \Bigg) (F_{SS'} \otimes \mathbbm{1}_{EE'})  \Bigg) \\
\qquad  &+ \Tr_{RR'} \Bigg( \Bigg( \frac{\mathbbm{1}_{RR'}}{d_{R}(d_{R}+1)} \Bigg) (\mathbbm{1}_{SS'} \otimes F_{EE'} )  \Bigg)\\ 
&\leq \Tr_{RR'} \Bigg( \Bigg( \frac{\mathbbm{1}_{R}}{d_{R}} \otimes  \frac{\mathbbm{1}_{R'}}{d_{R}}  \Bigg) (F_{SS'} \otimes \mathbbm{1}_{EE'})  \Bigg)\\
\qquad &+ \Tr_{RR'} \Bigg( \Bigg( \frac{\mathbbm{1}_{R}}{d_{R}} \otimes  \frac{\mathbbm{1}_{R'}}{d_{R}}  \Bigg) (\mathbbm{1}_{SS'} \otimes F_{EE'} )  \Bigg)
\end{align*}
	
Recordando que $\frac{\mathbbm{1}_{R}}{d_{R}}= \mathcal{E}_{R}$ y $\Omega_{S}=\Tr_{E}(\mathcal{E}_{R})$ se tiene que:

\begin{align*}
\Tr_{RR'}\Bigg(\frac{\mathbbm{1}_{R}}{d_{R}} \otimes \frac{\mathbbm{1}_{R'}}{d_{R}} (F_{SS'} \otimes 		\mathbbm{1}_{EE'}) \Bigg) &+ \Tr_{RR'}\Bigg(\frac{\mathbbm{1}_{R}}{d_{R}} \otimes \frac{\mathbbm{1}_{R'}}{d_{R}} (\mathbbm{1}_{SS'}\otimes 	F_{EE'}) \Bigg) \notag \\
&= \Tr_{SS'}((\Omega_{S} \otimes \Omega_{S})F_{SS'}) + \Tr_{EE'}((\Omega_{E} \otimes \Omega_{E})F_{EE'}).
\end{align*}
Entonces se llega a lo que se quería:	
\begin{equation}
\langle \Tr_{S} \rho_{S}^{2} \rangle \le \Tr_{S} \Omega_{S}^{2} +\Tr_{E} \Omega_{E}^{2},
\end{equation}
y esto es lo mismo que 
\begin{equation}
\langle \Tr_{S} \rho_{S}^{2} \rangle \le \Tr_{S} \langle \rho_{S} \rangle ^{2}  + \Tr_{E} \langle \rho_{E}.\rangle^{2}
\end{equation}

Gracias al resultado \ref{eq:0} se obtiene
\begin{equation}
\langle \norm{\rho_{S} -\Omega_{S}}_{1} \rangle \le \sqrt{d_{S}(Tr_{E} \langle \rho_{E} \rangle^{2})}.
\end{equation}
Si se define $d_{E}^{eff} \equiv \frac{1}{Tr_{E} \Omega_{E}^{2}}$ como la dimensión efectiva del ambiente en el estado canónico, esto mide la dimensión del espacio en el que el ambiente es más probable de estar, como $\langle \rho_{E} \rangle_{\phi} = \Omega_{E}$ se concluye que 
\begin{equation}
\langle \norm{\rho_{S} -\Omega_{S}}_{1} \rangle \le \sqrt{\frac{d_{S}}{d_{E}^{eff}}}
\end{equation}

Ya con esto se tiene el teorema \ref{teorema principal}. Cuando  el ambiente es mucho más grande que el sistema $\mu$ y $\mu'$  de la ecuación \ref{eq.3} serán pequeñas ($d_{E}^{eff}>>d_{S}$) implicando $\norm{\rho_{S} -\Omega_{S}}_{1} \approx 0$ con alta probabilidad. Aunque ya se llegó a la desigualdad que se quería se puede notar lo siguiente, sean los valores propios de $\Omega_{E}$ iguales a $\lambda_{E}^{k}$ con su máximo valor propio $\lambda_{E}^{max}$ se ve que 

\begin{align*}
\Tr_{E} \Omega_{E}^{2} &= \sum_{k} (\lambda_{E}^{k})\\
&\leq \lambda_{E}^{max} \sum_{k} \lambda_{E}^{k}\\
&= \max_{\ket{\phi_{E}}} \bra{\phi_{E}} \Tr_{S} \Big( \frac{\mathbbm{1}_{R}}{d_{R}} \Big) \ket{\phi_{E}}\\
&= \max_{\ket{\phi_{E}}} \sum_{s} \bra{s \phi_{E}} \frac{\mathbbm{1}_{R}}{d_{R}} \ket{s \phi_{E}} \\
&\leq \frac{d_{S}}{d_{E}}	
\end{align*}

en conclusión  $d^{eff}_{E} \geq d_{R}/ d_{S}$ entonces se obtiene

\begin{equation}
\langle \norm{\rho_{S} - \Omega_{S}}_{1} \rangle \leq \sqrt{\frac{d_{S}}{d_{E}^{eff}}} \leq  \sqrt{\frac{d_{S}^{2}}{d_{R}}}.
\end{equation}
Con esto se finaliza la demostración del teorema \ref{teorema principal} en la siguiente sección se hablará de sus consecuencias físicas.

\section{Significado físico}

El teorema anterior ya permite hablar del concepto importante que se sigue del aritculo de Popescu et al la idea general de la física es poder dar una relación uno a uno entre las propiedades de un objeto físico y su representación matemática con esta correspondencia se puede decir que la teoría esta completa cuando todas las propiedades que son posibles de medir tienen su semejante en la teoría \cite{Decoherence}. Por ejemplo en la física clásica a las cantidades como velocidad y distancia se le asigna los símbolos matemáticos $v$ y $x$, el cuerpo puede ser especificado dando estas dos cantidades en un tiempo determinado. La mecánica clásica nos da un ejemplo sencillo pero cuando se pasa a la mecánica estadística se encuentra perspectivas diferentes que entran en conflicto con esta sencilla idea. En los inicios de la mecánica estadística hubo muchas controversias dado a que ahora la idea determinista que Newton y otros habían mostrado como cierta (Se podía describir la naturaleza de manera predecible) ahora se introducía la probabilidad a sistemas que según Newton eran deterministas. En este punto se debe hacer claro que la necesidad de la probabilidad era por la falta de conocimiento pero no por un indeterminismo intrínseco como en la mecánica cuántica. Dos propuestas importantes surgieron en busca de una base conceptual sobre estas probabilidades, la visión de Gibbs y la de Boltzmann. Gibbs propuso la idea del ensamble, esta es una idea que hasta el día de hoy se sigue usando porque da los resultados correctos para las propiedades de sistemas termodinámicos.\\
La idea de Gibbs es la siguiente: Se tiene un sistema macroscópico con ciertos parámetros que se pueden medir. Este sistema está compuesto por varias partículas que tienen posiciones y momentos, pero estas propiedades están restringidas por los parámetros macroscópicos. Entonces como no se puede conocer el microestado del sistemas, no se puede medir cada posición y momento de cada partícula, se tiene una libertad sobre cuál es el microestado dado el macroestado. El ensamble es el conjunto de microestados que cumplen con el macroestado, mientras el microestado cumpla con los parámetros macroscópicos hará parte del ensamble así el sistema no se encuentre en ese microestado. En teoría no hay limitación sobre el número de microestados que pertenezcan a un ensamble, este puede tener infinitos microestados. Esta idea de base para la función de densidad en el espacio de fase, cada punto en este espacio es un microestado estos micro estados no interactuan entre ellos , con esto se puede construir la mecánica estadística que se conoce \cite{PathriaStat}. 
\\
Por el contrario Boltzmann da un concepción de la mecánica estadística más intuitiva; él propone $N$ partículas que interactúan entre ellas y cumplen los parámetros macroscópicos. Las partículas no son imaginarias como en el caso de los ensambles de Gibbs, ellas son las partículas que están en el sistema que se está estudiando. Se esperaría que la solución de Boltzmann fuera la que diera los resultados correctos dada la sencillez de su propuesta pero es la formulación de Gibbs la que da la termodinámica correcta. E.T Jaynes en \cite{JaynesEntropies} muestra las diferencias matemáticas de cada perspectiva y llega a la conclusión que la formulación de Gibbs da la entropía correcta sin importar cual sea el tipo de interacción que tenga el sistema, además encuentra que la entropía que sale de la formulación Boltzmanniana no es la misma entropía dada por la termodinámica,la entropía de Boltzmann es correcta cuando no hay interacción entre las partículas.
\\
Ya con la comprobación experimental se debería aceptar la perspectiva de Gibbs como la correcta y dejar a un lado la de boltzmann pero esto deja a la mecánica estadística en un contexto filosófico poco fuerte, aunque la física busque una correspondencia entre experimento y teoría no se puede aceptar cualquier teoría solo por tener congruencia con el experimento, la física debe buscar una estabilidad conceptual en sus teorías.¿ Por qué la perspectiva de Gibbs tiene un fundamento congruente ?  lo que propone Gibbs aunque matemáticamente aceptable se basa en todos los microestados en los que el macroestado no se encuentra es decir, que para poder determinar la propiedades de estado el cual se está estudiando se debe estudiar todos los estados en los que el sistema no se encuentra pero podría estar, pero por qué para analizar un sistema que se encuentra en un estado definido y exacto debería examinar los posibles infinitos estado en donde no se encuentra si se hace un promedio sobre estos microestados ¿ cuál es el significado de este promedio?. Este problema ontológico deja la perspectiva de Gibbs en duda, mientra la de Boltzmann habla del sistema que se examina sin ningún problema de este estilo. 
\\
La lucha entre las perspectiva ha sido una de aún no se resuelve lo que se ha mostrado hasta ahora ha sido la matemática de la tentativa de unir ambas perspectivas por Popescu et al. pero ¿ por qué estos resultado matemáticos intentan dar una luz a la resolución de estas perspectivas? se puede ver fácilmente que cuando se habla de del estado $ \Omega_{S} = \Tr_{B} \mathcal{E}_{R} $  este condensa la perspectiva de Gibbs mientras que $\rho_{S} = \Tr_{E} \ket{\phi} \bra{\phi}$ habla de un subsistema específico bajo una restricción esta es la perspectiva de Boltzmann por lo tanto el resultado mostrado acá dice: a menos que se tenga un estado bastante especial se puede asegurar que el subsistema de un universo muy grande, dado por un estado puro, se comporta de la misma manera que si el universo se encontrara en un estado máximamente mezclado. Esto corresponde a decir que para la mayoría de los estados en un espacio de Hilbert se puede reconciliar ambas perspectivas.
\\
Hay otro punto que da más mérito a la perspectiva mostrada acá y es que el estado $\Omega_{S}$ es la forma de mostrar la ignorancia subjetiva que se tiene del sistema. Se sabe que este es el punto de partida para la mecánica estadística pero esto también es un fundamento que generalmente es muy cuestionado gracias al teorema \ref{teorema principal} se cambia esa probabilidad subjetiva por una objetivad debido al entrelazamiento. Por este cambio a un fundamente más sólido no se debe entrar en el problema de ergodicidad ya que en general la solución del problema de esa probabilidad se dice que el sistema pasa por todos los estados,y eso solo complica la pregunta dado que hay sistemas que no pasan por todos los estados pero aún así son descritos por un estado canónico. Ya no debe entrarse en la pregunta sobre la ergodicidad del sistema dado que esta perspectiva muestra que los promedios no son necesarios en el momento de darle un fundamento a la mecánica estadística.




