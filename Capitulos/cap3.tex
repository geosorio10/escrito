\chapter{Evolución hacia el equilibrio}

Ya después de haber dado una explicación más sólida a los principios de la mecánica estadística la siguiente pregunta que debe responderse es,¿ cómo ocurre la termalización?. ¿ Esta termalización puede deducirse de las ecuaciones básicas ?. Este será la pregunta que Linden et al. quisieron atacar se procederá a mostrar sus resultados del artículo (Linden et al. ,2008). 
Una de tantas dificultades en la mecánica estadística es que sus principios se formulan desde la ignorancia subjetiva y los promedios de ensamble. Estos son principios físicos bastantes discutible.
Recientemente se ha dado cuenta que los promedio de ensamble y la ignorancia subjetiva no son necesarios, Porque sistemas cuánticos individuales pueden mostrar características estadísticas. Esto es debido al entrelazamiento lo cual es un efecto cuántico, esto hace que la falta de conocimiento ya no sea subjetiva sino objetiva. Por la teoría cuántica se sabe que aunque se tenga todo el sistema descrito por la función de onda, la cual da un conocimiento completo del sistema, los subsistemas de este pueden seguir en un estado desconocido. De forma más específica el sistema puede estar en un estado puro mientras que un subsistema puede estar en un estado mixto. En este caso no se puede conocer el subsistema ya que este se comporta como una densidad de probabilidad. La comparación con la contraparte clásica es sorprendente porque tener un conocimiento completo del sistema clásico es saber cualquier subsistema. Luego si se aplica una teoría clásica las probabilidades aparecen como falta del conocimiento posible de obtener, si se llega a conocer todo no abrían probabilidades.
Ya con resultados del capítulo anterior sobre: la mayoría de los estados puros de un sistema los subsistemas(suficientemente pequeños) están en un estado canónico. Este resultados aunque bastante general queda limitado por hablarse de un tiempo específico y habla de estados genéricos. Ahora se quiere ver su evolución temporal y en cuales circunstancias los sistemas llegan al equilibrio y fluctúan cerca a este. Los estados lejanos al equilibrio son no genéricos y serán tratados aquí. Para saber cómo moverse en este tema se va a determinar qué significa que un sistema esté termalizado. Para esto se darán las siguientes ideas:

\textbf{Equilibrio}: Se dice que un sistema se equilibra si este evoluciona a un estado específico(puede ser puro pero en general es mixto) y se mantiene allí por casi todo el tiempo. Dado esto no es importante cuál sea el estado de equilibrio este puede ser la distribución de Boltzmann o no. Además se puede relajar las condiciones de independencia del estado inicial, o sea puede depender del estado inicial del susbsistema y/o del estado inicial del ambiente de forma arbitraría. Esta definición de equilibrio es la parte más intuitiva a lo que se refiere sobre termalización. El hecho de que un estado se mantenga por un largo periodo de tiempo con las mismas características es lo que se piensa al  pensar en equilibrio. Aunque esto sería una forma general de hablar sobre equilibrio porque se da mucha libertad al estado y sus dependencias sobre el ambiente. Para seguir una idea más rigurosa de equilibrio se especifica lo sigeuinte.

\textbf{Independencia del ambiente}: El  estado de equilibrio del sistema no debería depender exactamente del estado inicial del baño. Esto quiere decir que el baño debe tener unos parámetros macroscópicos (como la temperatura) tales que al final llegan al equilibrio el estado dependa de la temperatura del baño. Con esto se puede restringir un poco más a lo que se llamará equilibrio.Esta idea también es proveniente de lo que normalmente se espera del equilibrio porque los parámetros macroscópicos son los que en general se tienen completamente especificados y se ve que para los mismos se tiene un mismo equilibrio. El estado exacto del baño no debería jugar un papel  tan importante ya que para los mismos parámetros macroscópicos pueden haber varios estados del baño que concuerden con ellos. Como se supone que con ciertos parámetros macroscópicos dados  un subsistema llegará al equilibrio sin importar cual ha sido su estado inicial. Pueden dos subsistemas preparados idénticamente haber sido iniciados en un estado A y el otro en el estado B, el equilibrio se obtendrá así hayan empezado de maneras diferentes. Esto motiva lo siguiente

\textbf{Independencia del estado del subsistema}:Si el subsistema es pequeño en comparación con el ambiente el estado de equilibrio del subsistema debería ser independiente de su estado inicial.

Pero para poder corroborar y unir estas ideas con los resultados ya bien conocidos se impone una última restricción

\textbf{Forma Boltzmanniana del estado de equilibrio}: Bajo condiciones del estado inicial y el hamiltoniano, el estado de equilibrio del subsistema puede ser escrito como $\rho_{S}= \frac{1}{Z} \exp(-\frac{H_{S}}{k_{B}T})$la forma familiar ya conocida.
Descomponiendo el problema de la termalización de esa forma permite ver cada uno de los aspectos por separado además de darle una generalidad a todo el tratamiento sin tener que restringirse a situaciones que usualmente se le asocian a la termalización. Por ejemplo no se debe quedar en el régimen de corta o débil interacción entre el sistema y el baño; decir que el baño es uno típico (Dado una temperatura o rango de energía), la energía puede llegar a ser una cantidad extensiva. Pueden tomarse situaciones en los que el sistema no llegue a equilibrio.
Se mostrará con suposiciones no muy fuertes que los primeros dos supuestos(Equilibrio e independencia del ambiente) son propiedades de sistemas cuánticos.



\textbf{El Hamiltoniano}: la evolución de todo el sistema viene dado por 
\begin{equation}
H= \sum_{k} E_{k} \ket{E_{k}} \bra{E_{k}},
\end{equation}
$\ket{E_{k}}$ es el estado propio con energía $E_{k}$.Al Hamiltoniano anterior se le dará la siguiente y única restricción: Que tenga brechas de energías no degeneradas. Esta restricción quiere decir que  un hamiltoniano tiene brechas de energía no degeneradas si para cualquier diferencia de energías propias que sea diferente acero determina los valores de energía involucrados. Por ejemplo si se tiene 4 valores propios de energía $E_{k}, E_{l},E_{m},E_{n}$ entonces $E_{k}-E_{l}=E_{m}-E_{n} $ implica $k=l$ y $m=n$, o $k=m$ y $l=n$. Esto implica que los niveles de energía no son iguales para diferentes estados (no son degenerados).
Esta restricción del Hamiltoniano implica que el subsistema y el baño no importa como se divida siempre van a estar interactuando. Esto excluye los hamiltonianos no interactuantes ($H=H_{S}+H_{E}$). Este tipo de hamiltonianos tienen muchas brechas de energía degeneradas. véase que si no hay interacción en el hamiltoniano la energía es $E=E_{S}+E_{E}$ sean $E_{1},E_{2},E_{3},E_{4}$ tales que se satisfaga $E_{i}=E_{i}^{S}+E_{i}^{E}$ $i=1,2,3,4$. Esto lleva a una brecha degenerada. Este supuesto no es tan fuerte como pueda llegar a parecer porque cualquier perturbación que se le haga al Hamiltoniano romperá las degeneraciones sin importar lo pequeña que seala perturbación. Aunque estos cambios se tarden en hacer efecto sobre la evolución del sistema las escalas temporales no son importantes en este momento. Con este supuesto se puede hablar de interacciones complejas que por lo general no son muy tratados como interacciones de larga distancia o interacciones entre todas las partículas esto hace que la energía no llegue a ser una cantidad extensiva.

\subsection{Equilibración}
El punto principal que se quiere dar a entender viene en forma del siguiente resultado: Para casa estado puro de un sistema cuántico que se compone por un número grande de estados de energía propios y el cual evoluciona bajo un Hamiltoniano que tiene brechas de energías no degeneradas y por lo demás arbitrario, es tal que cada subsistema pequeño llegará al equilibrio. Esto quiere decir que todos los subsistemas pequeños cumplirán con las ideas anteriores sobre equilibrio exactamente que el sistema evolucionará a un estado particular y se quedará cercano a él o en este durante la mayoría del tiempo.
Hay en este resultado un requerimiento que anteriormente no fue nombrado, el requerimiento de una cantidad grande de estados propios de energía. La necesidad de que el sistema tenga muchos estados propios de energía es equivalente a decir que el estado variará bastante durante su evolución temporal. viendo el caso trivial de un solo estado propio de energía es claro que este no cambiará para nada. Este caso tan particular no llega a ser de mucho interés porque en el sentido anterior de equilibrio no cambia para nada y se diría que este se encuentra ya en equilibrio. Los resultados que se quieren mostrar tomaŕan estados lejanos de equilibrio sistemas que no se encuentre en el pero aquí llegan las otras suposiciones hechas anteriormente para que el subsistema no dependa del estado inicial este debe perder la información entonces si el subsistema empieza lejano al equilibrio este pasará  por muchos estados en su camino al equilibrio lo cual implica que todo el sistema también evolucione en muchos estados. El hecho de que el subsistema haya llegado al equilibrio no significa que el sistema deje de evolucionar debido a la unitaridad este debe seguir evolucionando con la misma proporción de antes. Para que los estados en los que el subsistema se encuentre en no equilibrio ocurran poco, los estados del universo en los que el subsistema se encuentre en esas condiciones deben ser una fracción muy pequeña del total de estados por donde pasa el universo. Por esto el universo debe pasar por muchos estados y el requerimiento de que tenga muchos estados propios de energía se valida.
También se muestra que lo siguiente es suficiente: Cuando el estado de todo el sistema pasa por muchos estados diferentes cualquier estado pequeño del subsistema alcanza el equilibrio.
Otra forma de saber qué pasa con el sistema es observar qué ocurre con el ambiente. por unitaridad el sistema debe seguir evolucionando aunque el subsistema y se encuentre en equilibrio y no cambie. Esta Evolución puede darse por el cambio de correlaciones entre el subsistema y el baño o por cambios en el estado del baño. Lo que se muestra es: cuando el estado del baño pasa por muchos estados diferentes, cualquier subsistema alcanza el equilibrio.Con estas dos ideas se muestra que la equilibración ocurre en estados productos iniciales entre el subsistema y el baño, para casi todos los estados iniciales del baño.
La noción de evolución por muchos estados diferentes se puede escribir matemáticamente por la dimensión efectiva del estados promediado temporalmente $d^{eff}(\omega)$ donde $\omega=\langle \rho(t) \rangle_{t}$. La relación que se puede ver con los estados propios de energía es 
\begin{equation}
\ket{\psi(t)} = \sum_{k} c_{k} e^{-i\frac{E_{k}t}{\hbar}} \ket{E_{k}}
\end{equation}
luego el operador de densidad es 
\begin{equation}
\rho(t)=\sum_{k,l} c_{k} c_{l}^{*} e^{\frac{-i(E_{k}-E_{l})t}{\hbar}}\ket{E_{k}} \bra{E_{l}}
\end{equation}
entonces su promedio temporal es, recordando la condición de no degeneración de los niveles de energía
\begin{equation}
\omega= \sum_{k} |c_{k}|^{2} \ket{E_{k}} \bra{E_{k}}
\end{equation}
entoces la dimensión efectiva es 
\begin{equation}
d^{eff}(\omega)=\frac{1}{Tr(\omega^{2})}=\frac{1}{\sum_{k} |c_{k}|^{4} }
\end{equation}
De la misma manera el hechode que el baño pase por muchos estados diferentes viene dado por $d^{eff}(\omega_{B})$ por $\omega_{B}=\langle \rho(t) \rangle_{t}$. Como se espera que el baño al evolucionar pase por muchos más estados dado que el subsistema debe seguir evolucionando y el subsistema quede en un espacio de estados más pequeños se preve que $d^{eff}(\omega_{B})$ sea mucho más grande que $d_{S}$. Para formular ya el primer teorema se quiere ver la distancia entre $\rho_{S}(t)$ y su promedio temporal $\omega_{S}= \langle \rho_{S}(t) \rangle_{t}$. Como se espera que $\rho_{S}(t) $ vaya fluctuando alrededor de $\omega_{S}$ se analizará el promedio temporal de su distancia $\langle D(\rho_{S}(t) ,\omega_{S}) \rangle_{t}$ cuando este sea muy pequeño el subsistema debe pasar gran parte del tiempo muy cerca a $\omega_{S}$. Esto quiere decir que el subsistema se equilibrará(según la definición anterior) a $\omega_{S}$.

\textbf{Teorema 1}:Considere cualquier estado $\ket{\psi(t)} \in \mathcal{H}$ evolucionando bajo un Hamiltoniano con brechas de energía no-degeneradas. Luego la distancia promedio entre $\rho_{S}(t)$ y su promedio temporal $\omega_{S}$ está acotado por:

\begin{equation}
\langle D(\rho_{S}(t) ,\omega_{S}) \rangle_{t} \le \frac{1}{2} \sqrt{\frac{d_{S}}{d^{eff}(\omega-{B})}} \le \sqrt{\frac{d_{S}^{2}}{d^{eff}(\omega)}}.
\end{equation}
\textbf{Demostración}:Por la relación ya conocida entre la distanacia de traza y la norma de Hilbert-Schimidt que se usó en el capítulo anterior

\begin{equation}
\norm{M}_{1} \le \sqrt{n}\norm{M}_{2}
\end{equation}

se usa para la el operador $D(\rho_{1}, \rho_{2})$:

\begin{equation}
\frac{1}{2} Tr_{S} \sqrt{(\rho_{1} - \rho_{2})^{2}} \le \frac{1}{2}\sqrt{d_{S} Tr_{S} (\rho_{1}- \rho_{2})^{2}}
\end{equation}
Por la concavidad de la función raíz cuadrada

\begin{equation}
\langle D(\rho_{S}(t),\omega_{S}) \rangle_{t} \le \sqrt{  d_{S} \Big \langle Tr_{S}(\rho_{S}(t)-\omega_{S})^{2} \Big \rangle_{t}}
\end{equation}

usando las expansiones para $\rho_{S}$ y $\omega_{S}$
\begin{equation}
\rho_{S}(t)=\sum_{k,l} c_{k} c_{l}^{*} e^{\frac{-i(E_{k}-E_{l})t}{\hbar}} Tr_{B} (\ket{E_{k}} \bra{E_{l}}),
\end{equation}

\begin{equation}
\omega_{S} = \sum |c_{k}|^{2} Tr_{B}(\ket{E_{k}} \bra{E_{k}}),
\end{equation}

se puede escribir $\langle Tr_{S}(\rho_{S}(t)-\omega_{S})^{2} \rangle_{t}$ como 

\begin{equation}
\langle Tr_{S}(\rho_{S}(t)-\omega_{S})^{2} \rangle_{t}=\sum_{k \neq l} \sum_{m \neq n} \mathcal{T}_{klmn} Tr_{S} [Tr_{B}(\ket{E_{k}} \bra{E_{l}}) Tr_{B}(\ket{E_{k}} \bra{E_{l}})]
\end{equation}
donde $\mathcal{T}_{klmn}$ es :
\begin{equation}
\mathcal{T}_{klmn}=c_{k}c_{l}^{*}c_{m}c_{n}^{*} \Big \langle e^{\frac{-i(E_{k}-E_{l}+E_{m}-E_{n})t}{\hbar}} \Big \rangle_{t}
\end{equation}
debido a la restricción impuesta al Hamiltoniano de brechas de energías no degeneradas y como solo se toman elementos $k \neq l$ y $m\neq n$ los terminos que son diferentes de $0$ son $k=n$ y $l=m$.

Luego
 
\begin{multline}
\langle Tr_{S}(\rho_{S}(t)-\omega_{S})^{2} \rangle_{t} \\
	= \sum_{k \neq l} |c_{k}|^{2} |c_{l}|^{2} Tr_{S} [Tr_{B}(\ket{E_{k}} \bra{E_{l}}) Tr_{B}(\ket{E_{k}} \bra{E_{l}})] \\
	=\sum_{k \neq l} |c_{k}|^{2} |c_{l}|^{2} \sum_{ss' bb'} \bra{sb}\ket{E_{k}} \bra{E_{l}}\ket{s'b} \bra{s'b'}\ket{E_{l}} \bra{E_{k}}\ket{sb'} \\
	=\sum_{k \neq l} |c_{k}|^{2} |c_{l}|^{2} \sum_{ss' bb'} \bra{sb}\ket{E_{k}} \bra{E_{k}}\ket{sb'} \bra{s'b'}\ket{E_{l}} \bra{E_{l}}\ket{s'b}\\
	=\sum_{k \neq l} |c_{k}|^{2} |c_{l}|^{2} Tr_{B} [Tr_{S}(\ket{E_{k}} \bra{E_{k}}) Tr_{S}(\ket{E_{l}} \bra{E_{l}})] \\
	=\sum_{k \neq l}  Tr_{B} [Tr_{S}(|c_{k}|^{2}\ket{E_{k}} \bra{E_{k}}) Tr_{S}( |c_{l}|^{2} \ket{E_{l}} \bra{E_{l}})] 
\end{multline}
recordando la definición de $\omega_{B}$ y observando la segunda línea de la ecuación anterior
\begin{equation}
Tr_{B} \omega_{B}^{2} -\sum_{k} |c_{k}|^{2} Tr_{S}[ Tr_{B}( \ket{E_{k}} \bra{E_{k}})]^{2} \leq  Tr_{B}( \omega_{B}^{2}).
\end{equation}

Por la subaditividad débil de la entropía de Rényi:

\begin{equation}
Tr(\omega^{2}) \geq  \frac{Tr_{B}(\omega_{B}^{2})}{Rank(\rho_{S})} \geq \frac{Tr_{B}(\omega_{B}^{2})}{d_{S}}
\end{equation}

uniendo todos los resultados con la desigualdad inicial del promedio de la distancia

\begin{equation}
\langle D(\rho_{S}(t) ,\omega_{S}) \rangle_{t} \le  \frac{1}{2} \sqrt{d_{S} Tr_{B}(\omega_{B}^{2})} \le \frac{1}{2}\sqrt{d_{S}^{2}Tr(\omega^{2})}=\frac{1}{2}\sqrt{\frac{d_{S}^{2}}{d^{eff}(\omega)}}.
\end{equation}

Con este resultado se puede hablar de la termalización de una forma matemática consto se ve que el subsistema se equilibra cuando la dimensión de $d^{eff}(\omega)$ sea mucho mayor que la dimensión de dos copias del subsistema ($d_{S}^{2}$) o cuando la dimensión efectiva explorada por el baño $d^{eff}(\omega_{B})$ sea mucho más grande que la dimensión del subsistema.
EL resultado anterior tiene varias generalidades que se quiere recodar. La restricción impuesta sobre el Hamiltoniano  es una que no excluye muchos Hamiltonianos. Además de que esta ha sido la única restricción sobre todo el sistema , no se ha especificado nada del baño ni del subsistema. El baño no está necesariamente en equilibrio no se le ha dado ninguna interpretación térmica a ningún objeto tratado hasta ahora. no se ha hablado tampoco de ninguna forma en la que el subsistema llega al equilibrio ni que esta en algún estado específico.
Los valores propios de energía tampoco son importantes en las cotas dadas anteriormente, en el teorema al ser demostrado fue encontrado algunos valores propios de energía que al ser promediados dan 0. La energía es importante al buscar las formas exactas en las que evoluciona el sistema pero aquí se demostró que para la equilibración en intervalos de tiempo muy grandes no son muy importantes , las cotas son independientes del tiempo . La forma en que se dividió el sistema(subsistema y baño) solo es importante para el teorema 1 la dimensión del subsistema y no la especificación de la forma o un subsistema particular. Esto permite decir que cualquier subsistema con dimensión $d_{S}$ estará en equilibrio, los varios subsistemas bastante pequeños de dimensión $d_{S}$ también estarán en equilibrio.
\textbf{Equilibrio de conjuntos típicos}
Aunque se puso una cota a la fluctuación de $\rho_{S}(t)$ alrededor de $\omega_{S}$ se quiere ver cuales casos esta fluctuación es tan pequeña que el subsistema se equilibrará para esto se mostrará el siguiente teorema\\
\textbf{Terorema2}:
i)El promedio de la dimensión efectiva $\langle d^{eff}(\omega) \rangle_{\psi}$, donde el promedio es sobre estados puros aleatorios uniformemente distribuidos $\ket{\psi} \in \mathcal{H}_{R} \subset \mathcal{H}$. Es tal que 
\begin{equation}
\langle d^{eff}(\omega) \rangle_{\psi} \ge \frac{d_{R}}{2}
\end{equation}
ii)Para un estado aleatorio $\ket{\psi} \in \mathcal{H}_{R} \subset \mathcal{H}$, la probabilidad $Pr_{\psi} \{ d^{eff}(\omega) < \frac{d_{R}}{4}  \}$ de que $d^{eff}(\omega)$ es más pequeña que $\frac{d_{R}}{4}$ es:
\begin{equation}
Pr_{\psi} \{ d^{eff}(\omega) < \frac{d_{R}}{4}  \} \leq 2 \exp{-C \sqrt{d_{R}}}
\end{equation}
con constante $c= \frac{(ln 2)^{2}}{72 \pi^{3}} \approx 10^{-4}$.
La primera parte del teorema 2 nos habla de cómo el promedio de la dimensión efectiva es más grande que la dimensión del subespacio de Hilbert esto significa que si se tiene estados de un subespacio bastante grande se puede asegurar un $d^{eff}(\omega)$ grane que implica un $\langle D(\rho_{S}(t) ,\omega_{S}) \rangle_{t}$ pequeño. La segunda parte solo confirma de manera más estricta el hecho de encontrar un $d^{eff}(\omega)$ pequeño, mostrando que la probabilidad de encontrar una dimensión efectiva menor a $\frac{d_{R}}{4}$ es exponencialmente pequeña.\\
\textbf{equilibrio de estados genéricos}:
usando el teorema 2 para ver qué ocurre con un estado escogido de forma aleatorio del espacio total $\mathcal{H}$, un estado genérico. Esto es simplemente poner $d_{R}=d$ gracias a esto se comprende que $d^{eff}(\omega)~d$ ya que  hay una probabilidad exponencialmente baja para que se dé  el caso en que $d^{eff} < \frac{d}{4}$.Dado que $d=d_{S}d_{B}$ la cota para las fluctuaciones queda $\sqrt{\frac{d_{S}}{d_{B}}}$ para un sistema de muchas partículas la dimensión del espacio de Hilbert crece de manera exponencial, si el subsistema es una fracción constante del número de partículas del baño esta proporción caerá de manera exponencial con el número total de de partículas luego los subsistemas se equilibrarán.\\
\textbf{Equilibración de sistemas lejanos del equilibrio}:
Qué ocurre con los sistemas que están lejos del equilibrio. Lo dicho arriba ya cualquier se pensaría que  que cualquier sistema debe llegar al equilibrio pero esto no es cierto debido a que arriba se usó un estado genérico y los estados lejos del equilibrio no son genéricos; los estados lejos del equilibrio no son típicos por el capítulo anterior se sabe que cono cotas exponenciales la mayoría de los estados en el espacio de Hilbert son tales que un subsistema pequeño está en un estado canónico. Para poder sacar algo de esta pregunta se planteará la situación normal, hay un baño que consiste de un número muy grande de partículas de las cuales se conoce unos parámetros macroscópicos, dentro de este se pone un subsistema con un estado inicial arbitrario pero descorrelacionado con el ambiente. Ahora la pregunta es ¿ el subsistema se equilibra?, se verá que para cualquier estado inicial del subsistema y para casi todos los estados iniciales del baño el subsistema se equilibra. Esto incluye cuando el subsistema está lejos del equilibrio.
El estado inicial del sistema está dado por $\ket{\Psi}_{SB}=\ket{\psi}_{S} \ket{\psi}_{B}$. El estado del subsistema es uno arbitrario $\ket{\psi}_{S}$ en el espacio de Hilbert. Dado unos parámetros macroscópicos el baño  debe cumplir con estos luego el estado del baño $\ket{\phi}_{B} \in \mathcal{H}_{B}^{R} \subset\mathcal{H}_{B}$. Esta restricción mantiene la generalidad de seguir en cualquier espacio de Hilbert restringido. Puede tener o no un sentido termodinámico o macroscópico. Pero la restricción es solo inicial al evolucionar el baño en el tiempo este puede moverse fuera de $\mathcal{H}_{B}^{R}$. Usando el teorema 2 para $\mathcal{H}_{R}= \ket{\psi}_{S} \otimes \mathcal{H}_{B}^{R}$ luego $d_{R}=d_{B}^{R}$ esto da como resultado que para casi todos los estados iniciales del baño y cualquier estado del subsistema $d^{eff}(\omega) \geq \frac{d_{B}^{R}}{4}$ lo cual significa que el subsistema se equilibrará para esas condiciones, mientras que $d_{B}^{R} >> d_{S}^{2}$.

El mecanismo en que el subsistema se equilibra puede ser bastante complicada ya que el baño pasa por muchos estados diferentes y no llega el equilibrio. Aunque el baño no llegue a hacerlo y se salga del subespacio $\mathcal{H}_{B}^{R}$ y el subsistema  puede equilibrarse de todas formas. En principio puede que la evolución del subsistema sea sensible a la forma precisa del baño.
Para ver que el baño no se equilibra de manera genérica se verá que $d^{eff}(\omega_{S})$ es mucho mayor que $d^{eff}(\rho_{B}(t))$ lo cual muestra que el baño sigue evolucionando y no se equilibra en ningún estado. Como los dos sistemas están en un estado puro $Rank(\rho_{B}(t))=Rank(\rho_{S}(t)) \geq d_{S}$ como la dimensión efectiva de un estado es siempre menor a su rango se obtiene 

\begin{equation}
d^{eff}(\rho_{B}(t)) \geq d_{S}
\end{equation}

pero 

\begin{equation}
d^{eff}(\omega_{B})\leq \frac{d_{eff}(\omega)}{d_{S}}
\end{equation}

Pero para un estado genérico el teorema 2 ii dice $d^{eff}(\omega) > \frac{d_{R}}{4}$ se tiene

\begin{equation}
d^{eff}(\omega_{B}) \leq \frac{d^{R}}{4d_{S}}=\frac{d_{B}^{R}}{4d_{S}} >>d_{S} \leq d^{eff}(\rho_{B}(t))
\end{equation}
\textbf{Independencia del estado inicial}
LO que se ha mostrado hasta ahora ha sido como si el subsistema es pequeño este se equilibrará. Ahora el enfoque que se quiere mostrar es cual sería la dependencia del estado de equilibrio del subsistema. Hasta ahora el estado inicial podría hacer que el equilibrio sea un estado diferente dependiendo del inicial. Sea el estado de equilibrio del subsistema $\omega_{S}^{\psi}$.Como es conocido se desearía que el estado de equilibrio dependa solo de los parámetros macroscópicos y no del estado inicial microscópico.
El teorema siguiente prueba que para casi todos los estados en un subsistema restringido llevan al mismo estado de equilibrio.
\textbf{Teorema 3}:
i) Casi todos los estados iniciales de un subespacio restringido bastante grande llevan al mismo estado de equilibrio de un subsistema pequeño. En particula, con $\langle .\rangle_{\psi}$ siendo el promedio sobre estados puros aleatoriamente uniformes $\ket{\psi(0)} \in \mathcal{H}_{R} \subset \mathcal{H}_{S} \otimes \mathcal{H}_{B}$ u $\Omega_{S}= \langle \omega_{S}^{\psi} \rangle_{\psi}$ :
\begin{equation}
\langle D(\omega_{S}^{\psi}, \Omega_{S}) \rangle_{\psi} \leq \sqrt{\frac{d_{S} \delta}{4d_{R}}} \leq \sqrt{\frac{d_{S}}{4d_{R}}}
\end{equation}
La primera desigualdad es más estricta pero más complicada 
\begin{equation}
\delta= \sum_{k}\bra{E_{k}}\frac{\Pi_{R}}{d_{R}}\ket{E_{k}} Tr_{S} \Big( Tr_{B} (\ket{E_{k}} \bra{E_{k}}) \Big)^{2} \leq  1
\end{equation}
Donde $\Pi_{k}$ es el proyector sobre $\mathcal{H}_{R}$
ii)Paraunestado aleatorio $\ket{\psi} \in \mathcal{H}_{R}\subset \mathcal{H}$, la probabilidad que $D(\omega_{S}^{\psi},\Omega_{S}) > \frac{1}{2} \sqrt{\frac{d_{S} \delta}{d_{R}}}+ \epsilon$caiga exponencialmente con $\epsilon^{2}d_{R}$:
\begin{equation}
Pr_{\psi} \{ D(\omega_{S}^{\psi}, \Omega_{S}) > \frac{1}{2}\sqrt{\frac{d_{S} \delta}{d_{R}}} + \epsilon \} \leq 2exp(-C'\epsilon^{2}d_{R}),
\end{equation}
con $C'=\frac{2}{9 \pi^{3}}$ si se pone $\epsilon=d_{R}^{-1/3}$ da una distancia promedio pequeña con alta probabilidad cuando $d_{R}>>d_{S}$.

\textbf{Independencia del estado del ambiente}
PAra un estado inicial que sea el producto del estado del sistema y el del ambiente $\ket{\psi}_{SB}= \ket{\psi}_{S} \ket{\phi}_{B}$ en el espacio $\mathcal{H}_{R}=\ket{\psi} \otimes \mathcal{H}_{R}^{B}$.ya se mostró que para estados genéricos el ambiente causa que el subsistema se equilibre aunque el estado de equilibrio $\omega_{S}^{\psi}$ del subsistema podría  depender del estado inicial del baño $\ket{\phi}_{B}$. Esto no es así y casi todos los estados del baño en $\mathcal{H}_{B}^{R}$ llevan al mismo estado de equilibrio del subsistema simplemente usando el resultado anterior a  $\mathcal{H}_{B} = \ket{\psi}\otimes \mathcal{H}_{R}^{B}$ y luego $d_{R} = d_{R}^{B}$, si $d_{R}^{B}>>d_{S}$ para casi todos los estados iniciales del baño llevarán al subsistema al mismo estado de equilibrio $\omega_{S}$. como se usó la cota menos estricta este resultado no depende de la forma explícita de los estados de energías propios.
\textbf{Independenciadelestado del sistema}: Como yase hadicho el subsistema en contacto con el baño llega al equilibrio sin importar su estado inicial pero sí depende del baño. Este resultado no se pudo resolver pero se da más indicios.
Los problemas son que el estado de equilibrio no siempre es independiente del estado inicial del subsiste,a. si el subsistema puede cambiar al ambiente de forma drástica el estado de equilibrio sí dependerá del estado inicial del subsistema. aquí muestra que las dimensiones del espacio de Hilbert del subsistemas el ambiente son importante pero el valor se le haya dado una restricción al Hamiltoniano (Brechas de energías no degeneradas ) esta condición no están estricta como para no permitir cantidades conservadas en el subsistema cuando estas cantidades conservadas existen los estados del subsistema con diferentes constantes(estados inicial es diferentes )no podrán llegar al mismo estado de equilibrio.


