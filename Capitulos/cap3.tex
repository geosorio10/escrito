\chapter{Evolución Hacia El Equilibrio}

En el capítulo anterior se expuso la idea de Popescu et al. \cite{Popescu2006} para poder reconciliar las ideas fundamentales de la mecánica estadística. Se mostró que para estados genéricos de un universo se tiene que el estado del sistema va a estar muy cercano del estado canónico. Ya con esta idea la siguiente pregunta que se puede hacer es sobre la termalización, ¿ Este fenómeno cómo ocurre? ¿ se puede deducir se las ecuaciones básicas(Schrödinger, Newton , etc)?. Este capítulo mostrará lo que Linden et al. \cite{LindenPaper} proponen para poder responder estas preguntas siguiendo las ideas ya dadas en el capítulo anterior. La generalidad que se ha dado a los fundamentos gracias Popescu et al. permite una flexibilidad al momento de manejar nuevos problemas por eso se ve que además de tener una buena base filosófica las matemáticas dan un buen esqueleto para sostener las bases y seguir construyendo con ellas. Ahora se quiere tratar el caso en que el estado no se encuentra en un tiempo específico sino ya en un tiempo  general. Entonces se deberá apuntar la maquinaria que ya se tiene para poder redondear el problema; esto será lo primero a  precisar. Luego se mostrará que por lo general los estados del sistema llegan al equilibrio y duran allí gran parte de su tiempo, todo esto bajo las especificaciones dadas al comienzo. Se seguirá dando un teorema que muestra como para la gran mayoría de casos de estados genéricos se llegará al equilibrio, aquí también se dará un argumento sobre cómo los estados lejanos al equilibrio  llegan a equilibrarse. Además se mostrará la independencia del estado inicial y problemas que no se han llegado a solucionar desde esta perspectiva.

\section{Especificación del Equilibrio}

Las preguntas planteadas anteriormente quieren enfocarse en el equilibrio pero para poder seguir viendo qué ocurre   con este fenómeno se deberá establecer lo que significa, basados en conceptos fenomenológicos se dará la  definición del equilibrio y qué se espera de este estado.
\\
La manera más intuitiva que se tiene sobre el equilibrio es que el sistema, que se está analizando, se mantiene en el mismo estado durante un largo periodo de tiempo con las mismas características; entonces se dice que un sistema se equilibra si este evoluciona a un estado específico (puede ser puro pero en general es mixto) y se mantiene allí por casi todo el tiempo. Aún no se dice nada  sobre las dependencias que pueda tener, o sea se puede tener una condición de dependencia laxa esto permite decir que el estado de equilibrio depende o no del estado inicial del susbsistema y/o del estado inicial del ambiente de forma arbitraría. Dado esto no es importante cuál sea el estado de equilibrio este puede ser la distribución de Boltzmann o cualquier otra. El concepto de equilibrio dicho aquí es una forma general de hablar sobre este fenómeno porque se da mucha libertad al estado y sus dependencias sobre el ambiente por eso se seguirá puliendo para no dejarlo tan general.
\\
El  estado de equilibrio del sistema no debería depender exactamente del estado inicial del baño. Esto quiere decir que el baño debe tener unos parámetros macroscópicos (como la temperatura) tales que al cuando se llega al equilibrio el estado dependa de la de estos parámetros, ya se restringe un poco más a lo que se llamará equilibrio.Esta idea también es proveniente de lo que normalmente se espera del equilibrio porque los parámetros macroscópicos son los que en general se tienen completamente especificados y estos establecen el equilibrio. El estado exacto del baño no debería jugar un papel  tan importante porque para los mismos parámetros macroscópicos pueden haber varios estados del baño que concuerden con ellos, se supone que con ciertos parámetros macroscópicos dados  un subsistema llegará al equilibrio sin importar cual ha sido su estado inicial. Pueden dos subsistemas preparados con los mismos parámetros macroscópicos haber sido iniciados en un estado A y el otro en el estado B llegarán al mismo estado de equilibrio sin importar en cual de los dos hayan empezado.  Esto motiva lo siguiente:
Si el subsistema es pequeño en comparación con el ambiente el estado de equilibrio del subsistema debería ser independiente de su estado inicial. Pero para poder corroborar y unir estas ideas con los resultados conocidos se impone una última restricción. Bajo condiciones del estado inicial y el Hamiltoniano, el estado de equilibrio del subsistema puede ser escrito como $\rho_{S}= \frac{1}{Z} \exp(-\frac{H_{S}}{k_{B}T})$ la forma familiar dada por la mecánica estadística.
\\
Haber podido dividir el problema de la termalización de esta forma permite analizar cada uno de los aspectos por separado además de darle una generalidad a todo el tratamiento sin tener que restringirse a situaciones que usualmente se le asocian a la termalización. Por ejemplo no se debe quedar en el régimen de corta o débil interacción entre el sistema y el baño o decir que el baño es uno típico (Dado una temperatura o rango de energía). Pueden tomarse situaciones en los que el sistema no llegue a equilibrio. 
\\
Lo siguiente es que se mostrará con suposiciones no muy fuertes los primeros dos supuestos, la idea de que un sistema llegará al equilibrio y que el estado no depende del ambiente, son propiedades de sistemas cuánticos.La evolución del sistema tiene implicado al Hamiltoniano por eso se estudiará el siguiente:
\begin{equation}
H= \sum_{k} E_{k} \ket{E_{k}} \bra{E_{k}},
\end{equation}
donde $\ket{E_{k}}$ es el estado propio con energía $E_{k}$. Al Hamiltoniano anterior se le dará la única restricción de que tenga brechas de energías no degeneradas. Esto quiere decir que para una diferencia de energías dada, solo existe un par posible de estados con esa diferencia. El ejemplo puesto por (Linden et al.) es: si se tiene 4 valores propios de energía $E_{k}, E_{l},E_{m},E_{n}$ entonces $E_{k}-E_{l}=E_{m}-E_{n} $ implica $k=l$ y $m=n$, o $k=m$ y $l=n$. Esto implica que los niveles de energía no son iguales para diferentes estados (no son degenerados).\\
Esta restricción del Hamiltoniano comprende que no importa como se divida el subsistema y el baño  siempre van a estar interactuando, esto excluye los Hamiltonianos no interactuantes ($H=H_{S}+H_{E}$) estos Hamiltonianos tienen muchas brechas de energía degeneradas. Véase que si no hay interacción en el Hamiltoniano la energía es $E=E_{S}+E_{B}$ y sean $E_{1},E_{2},E_{3},E_{4}$ tales que se satisfaga $E_{i}=E_{i}^{S}+E_{i}^{E}$ $i=1,2,3,4$, esto lleva a una brecha degenerada. Este supuesto no es tan fuerte como pueda llegar a parecer porque cualquier perturbación que se le haga al Hamiltoniano romperá las degeneraciones sin importar lo pequeña que sea la perturbación. Aunque estos cambios se tarden en hacer efecto sobre la evolución del sistema las escalas temporales no son importantes en este momento.  Gracias a esta restricción se pueden incluir interacciones complejas que por lo general no son analizadas en la literatura como interacciones de larga distancia o interacciones entre todas las partículas esto hace que la energía no llegue a ser una cantidad extensiva.

\section{Equilibración}

La perspectiva principal que se quiere dar a entender siguiendo a \cite{LindenPaper} viene en forma del siguiente resultado: Para cada estado puro de un sistema cuántico que se compone por un número grande de estados de energía propios y el cual evoluciona bajo un Hamiltoniano que tiene brechas de energías no degeneradas y por lo demás arbitrario, es tal que cada pequeño subsistema llegará al equilibrio. Esto quiere decir que todos los subsistemas pequeños cumplirán con las ideas anteriores sobre equilibrio exactamente que el sistema evolucionará a un estado particular y se quedará cercano a él o en este durante la mayoría del tiempo.
\\
Hay en este resultado un requerimiento que anteriormente no fue nombrado, el requerimiento de una cantidad grande de estados propios de energía. La necesidad de que el sistema tenga muchos estados propios de energía es equivalente a decir que el estado variará bastante durante su evolución temporal, por ejemplo el caso trivial de un solo estado propio de energía es claro que este no cambiará para nada. Este caso tan particular no llega a ser de mucho interés porque este estado no evolucionará a otro y se diría que ya se encuentra en equilibrio. Los resultados que se quieren mostrar tomaran estados fuera del equilibrio. En este punto se usan las otras suposiciones hechas anteriormente para que el subsistema no dependa del estado inicial el subsistema debe perder esta información; si el subsistema empieza lejano al equilibrio este pasará  por muchos estados en su camino al equilibrio lo cual implica que el universo también evolucione en muchos estados. El hecho de que el subsistema haya llegado al equilibrio no significa que el universo deje de evolucionar, debido a la unitaridad, debe seguir evolucionando con la misma proporción que antes. Para que los estados en los que el subsistema se encuentre en no equilibrio ocurran poco, los estados del universo en los que el subsistema se encuentre en esas condiciones deben ser una fracción muy pequeña del total de estados por donde pasa el universo. Por esto el universo debe pasar por muchos estados y el requerimiento de que tenga muchos estados propios de energía se valida. 
\\
Otra forma de saber qué pasa con el universo es observar qué ocurre con el ambiente. Por unitaridad el universo debe seguir evolucionando aunque el subsistema y se encuentre en equilibrio y no cambie. Esta Evolución puede darse por el cambio de correlaciones entre el subsistema y el baño o por cambios en el estado del baño. Lo que se muestra es: cuando el estado del baño pasa por muchos estados diferentes, cualquier subsistema alcanza el equilibrio. También se muestra que cuando el estado del universo pasa por muchos estados diferentes cualquier estado pequeño subsistema alcanza el equilibrio. Estas dos ideas  exponen que la equilibración ocurre en estados productos iniciales entre el subsistema y el baño, para casi todos los estados iniciales del baño.
\\

Para poder empezar a construir estos conceptos matemáticamente se puede empezar con la concepción matemática de evolucionar por muchos estados diferentes, esta se comprime en la dimensión efectiva del estado promediado temporalmente $d^{eff}(\omega)$ donde $\omega= \big \langle \rho(t) \big\rangle_{t}$. Esta medida de evolución por muchos estados se puede relacionar con los estados propios de energía así:
\\
Se toma el estado del universo
\begin{equation}
\ket{\psi(t)} = \sum_{k} c_{k} e^{-i\frac{E_{k}t}{\hbar}} \ket{E_{k}}
\end{equation}
cuyo operador de densidad es 
\begin{equation}
\rho(t)=\sum_{k,l} c_{k} c_{l}^{*} e^{\frac{-i(E_{k}-E_{l})t}{\hbar}}\ket{E_{k}} \bra{E_{l}},
\end{equation}
su promedio temporal es, recordando la condición de no degeneración de los niveles de energía,
\begin{equation}
\omega= \sum_{k} |c_{k}|^{2} \ket{E_{k}} \bra{E_{k}}
\end{equation}
y dando la relación se calcula la dimensión efectiva resultando 
\begin{equation}
d^{eff}(\omega)=\frac{1}{Tr(\omega^{2})}=\frac{1}{\sum_{k} |c_{k}|^{4} }.
\end{equation}
\\
Similarmente el hecho de que el baño pase por muchos estados diferentes viene dado por $d^{eff}(\omega_{B})$ con $\omega_{B}=\langle \rho(t) \rangle_{t}$. Debido a que el baño al evolucionar pase por muchos más estados dado que el universo debe seguir evolucionando y el subsistema quede en un espacio de estados más pequeño se preve que $d_{S}$ es mucho más pequeño que $d^{eff}(\omega_{B})$. Para formular ya el primer teorema se quiere ver la distancia entre $\rho_{S}(t)$ y su promedio temporal $\omega_{S}= \langle \rho_{S}(t) \rangle_{t}$. Como se espera que $\rho_{S}(t) $ vaya fluctuando alrededor de $\omega_{S}$ se analizará el promedio temporal de su distancia $\langle D(\rho_{S}(t) ,\omega_{S}) \rangle_{t}$ cuando este sea muy pequeño el subsistema debe pasar gran parte del tiempo muy cerca a $\omega_{S}$. Esto quiere decir que el subsistema se equilibrará(según la definición anterior) a $\omega_{S}$.

\begin{theorem} \label{equlibracion}

Considere cualquier estado $\ket{\psi(t)} \in \mathcal{H}$ evolucionando bajo un Hamiltoniano con brechas de energía no-degeneradas. Luego la distancia promedio entre $\rho_{S}(t)$ y su promedio temporal $\omega_{S}$ está acotado por:

\begin{equation}
\langle D(\rho_{S}(t) ,\omega_{S}) \rangle_{t} \le \frac{1}{2} \sqrt{\frac{d_{S}}{d^{eff}(\omega_{B})}} \le \frac{1}{2} \sqrt{\frac{d_{S}^{2}}{d^{eff}(\omega)}}.
\end{equation}
\end{theorem}

\textbf{Demostración}:
Recordando la relación entre la distancia de traza y la norma de Hilbert-Schmidt que se usó en el capítulo anterior

\begin{equation}
\norm{M}_{1} \le \sqrt{n}\norm{M}_{2},
\end{equation}

se usa para el operador $D(\rho_{1}, \rho_{2})$:

\begin{equation}
\frac{1}{2} \Tr_{S} \sqrt{(\rho_{1} - \rho_{2})^{2}} \le \frac{1}{2}\sqrt{d_{S} \Tr_{S} (\rho_{1}- \rho_{2})^{2}},
\end{equation}

por la concavidad de la función raíz cuadrada se obtiene

\begin{equation}
\langle D(\rho_{S}(t),\omega_{S}) \rangle_{t} \le \sqrt{  d_{S} \Big \langle \Tr_{S}(\rho_{S}(t)-\omega_{S})^{2} \Big \rangle_{t}}.
\end{equation}

Usando las expansiones para $\rho_{S}$ y $\omega_{S}$
\begin{equation}
\rho_{S}(t)=\sum_{k,l} c_{k} c_{l}^{*} e^{\frac{-i(E_{k}-E_{l})t}{\hbar}} \Tr_{B} (\ket{E_{k}} \bra{E_{l}}),
\end{equation}

\begin{equation}
\omega_{S} = \sum |c_{k}|^{2} \Tr_{B}(\ket{E_{k}} \bra{E_{k}}),
\end{equation}

se puede escribir $ \big \langle \Tr_{S}(\rho_{S}(t)-\omega_{S})^{2} \big \rangle_{t}$ como 

\begin{equation}
\big \langle \Tr_{S}(\rho_{S}(t)-\omega_{S})^{2}  \big \rangle_{t}=\sum_{k \neq l} \sum_{m \neq n} \mathcal{T}_{klmn} \Tr_{S} [\Tr_{B} \ket{E_{k}} \bra{E_{l}} \Tr_{B} \ket{E_{k}} \bra{E_{l}}]
\end{equation}
donde $\mathcal{T}_{klmn}$ es :
\begin{equation}
\mathcal{T}_{klmn}=c_{k}c_{l}^{*}c_{m}c_{n}^{*} \Big \langle e^{\frac{-i(E_{k}-E_{l}+E_{m}-E_{n})t}{\hbar}} \Big \rangle_{t}.
\end{equation}
Debido a la restricción impuesta al Hamiltoniano de brechas de energías no degeneradas y como solo se toman elementos $k \neq l$ y $m\neq n$ los terminos que son diferentes de $0$ son $k=n$ y $l=m$,

entonces
 
\begin{align*}
\big \langle \Tr_{S}(\rho_{S}(t)-\omega_{S})^{2} \big \rangle_{t} &= \sum_{k \neq l} |c_{k}|^{2} |c_{l}|^{2} \Tr_{S} [\Tr_{B}(\ket{E_{k}} \bra{E_{l}}) \Tr_{B}(\ket{E_{k}} \bra{E_{l}})]\\
	&=\sum_{k \neq l} |c_{k}|^{2} |c_{l}|^{2} \sum_{ss' bb'} \bra{sb}\ket{E_{k}} \bra{E_{l}}\ket{s'b} \bra{s'b'}\ket{E_{l}} \bra{E_{k}}\ket{sb'}\\
	&=\sum_{k \neq l} |c_{k}|^{2} |c_{l}|^{2} \sum_{ss' bb'} \bra{sb}\ket{E_{k}} \bra{E_{k}}\ket{sb'} \bra{s'b'}\ket{E_{l}} \bra{E_{l}}\ket{s'b}\\
	&=\sum_{k \neq l} |c_{k}|^{2} |c_{l}|^{2} \Tr_{B} [\Tr_{S}(\ket{E_{k}} \bra{E_{k}}) \Tr_{S}(\ket{E_{l}} \bra{E_{l}})]\\
	&=\sum_{k \neq l}  \Tr_{B} [\Tr_{S}(|c_{k}|^{2}\ket{E_{k}} \bra{E_{k}}) \Tr_{S}( |c_{l}|^{2} \ket{E_{l}} \bra{E_{l}})]\\
&=\Tr_{B} \omega_{B}^{2} -\sum_{k} |c_{k}|^{4} \Tr_{S}[ \Tr_{B}( \ket{E_{k}} \bra{E_{k}})]^{2} \leq  \Tr_{B}( \omega_{B}^{2}).
\end{align*}

La última  igualdad se encuentra recordando la definición de $\omega_{B}$ y observando la segunda igualdad. Por la subaditividad débil de la entropía de Rényi \cite{RenyiEntropia}:

\begin{equation}
\Tr(\omega^{2}) \geq  \frac{\Tr_{B}(\omega_{B}^{2})}{ \rank (\rho_{S})} \geq \frac{\Tr_{B}(\omega_{B}^{2})}{d_{S}}.
\end{equation}

Uniendo todos los resultados con la desigualdad inicial del promedio de la distancia

\begin{equation}
\langle D(\rho_{S}(t) ,\omega_{S}) \rangle_{t} \le  \frac{1}{2} \sqrt{d_{S} \Tr_{B}(\omega_{B}^{2})} \le \frac{1}{2}\sqrt{d_{S}^{2} 	\Tr(\omega^{2})}=\frac{1}{2}\sqrt{\frac{d_{S}^{2}}{d^{eff}(\omega)}}.
\end{equation}

Este resultado  da base para hablar de la termalización de una forma matemática, se ve que el subsistema se equilibra cuando la dimensión de $d^{eff}(\omega)$ sea mucho mayor que la dimensión de dos copias del subsistema ($d_{S}^{2}$) o cuando la dimensión efectiva explorada por el baño $d^{eff}(\omega_{B})$ sea mucho más grande que la dimensión del subsistema.
\\
EL resultado anterior tiene varias generalidades que se quieren recodar. La restricción impuesta sobre el Hamiltoniano  es una que no excluye muchos Hamiltonianos \cite{SakuraiQuantum}. Además de que esta ha sido la única restricción sobre todo el universo, no se ha especificado nada del baño ni del subsistema. El baño no está necesariamente en equilibrio no se le ha dado ninguna interpretación con respecto a la termodinámica usual a ningún objeto tratado hasta ahora. no se ha hablado tampoco de ninguna forma en la que el subsistema llega al equilibrio ni que esta en algún estado específico.
\\
Los valores propios de energía tampoco son importantes en las cotas dadas anteriormente, en el teorema al ser demostrado fue encontrado algunos valores propios de energía que al ser promediados dan 0. La energía es importante al buscar las formas exactas en las que evoluciona el sistema pero aquí se demostró que para la equilibración en intervalos de tiempo muy grandes no son muy importantes, las cotas son independientes del tiempo. La forma en que se dividió el universo (subsistema y baño) no es importante, solo es importante para el teorema \ref{equlibracion} la dimensión del subsistema y no la especificación de la forma o un subsistema particular. Esto permite decir que cualquier subsistema con dimensión $d_{S}$ estará en equilibrio, los varios subsistemas bastante pequeños de dimensión $d_{S}$ también estarán en equilibrio.
\\
El teorema \ref{equlibracion} puso una cota a la fluctuación de $\rho_{S}(t)$ alrededor de $\omega_{S}$ esto ya es un inicio pero aunque esta cota exista no se ha hablado de cuales sistemas tendrán fluctuaciones suficientemente pequeñas para poder decir que se encuentra en equilibrio. Ahora se explorará cuáles son los estados que se equilibrarán, el siguiente teorema dirá cuales estados tienen fluctuaciones muy pequeñas.

\begin{theorem}\label{teorema2}

\begin{enumerate}

\item El promedio de la dimensión efectiva $\langle d^{eff}(\omega) \rangle_{\psi}$, donde el promedio es sobre estados puros aleatorios uniformemente distribuidos $\ket{\psi} \in \mathcal{H}_{R} \subset \mathcal{H}$. Es tal que
\begin{equation}
\langle d^{eff}(\omega) \rangle_{\psi} \ge \frac{d_{R}}{2}.
\end{equation}
\item Para un estado aleatorio $\ket{\psi} \in \mathcal{H}_{R} \subset \mathcal{H}$, la probabilidad $\Pr_{\psi} \{ d^{eff}(\omega) < \frac{d_{R}}{4}  \}$ de que $d^{eff}(\omega)$ es más pequeña que $\frac{d_{R}}{4}$ es exponencialmente pequeña:
\begin{equation}
Pr_{\psi} \{ d^{eff}(\omega) < \frac{d_{R}}{4}  \} \leq 2 \exp{-C \sqrt{d_{R}}},
\end{equation}
con constante $c= \frac{(\ln 2)^{2}}{72 \pi^{3}} \approx 10^{-4}$.
\end{enumerate}
\end{theorem}

\textbf{Demostración}

\begin{enumerate}
\item
Primero se prueba la cota de la pureza esperada de $\omega$. Para esta prueba se usarán varios resultados ya empleados en el capítulo anterior como: la operación "flip"  $F$ y la identidad $\langle \ket{\psi} \bra{\psi} \otimes \ket{\psi} \bra{\psi} \rangle_{\psi} =  \frac{\Pi_{RR}(\mathbbm{1}+F)}{d_{R}(d_{R}+1)}$. Además se usará la notación de $ \ket{E_{k}} \equiv \ket{k}$ y $\ket{E_{k}} \otimes \ket{E_{l}} \equiv \ket{kl}$. Se introduce el mapeo de desfase como $ \$[ \rho ] \equiv  \sum_{k} \ket{k} \bra{k} \rho \ket{k}\bra{k}$, implica que $\omega= \langle \ket{\psi}\bra{\psi} \rangle_{t}= \$ [\ket{\psi}\bra{\psi}]$. Entonces usando la operación "flip"  se escribe $\langle \Tr (\omega)^{2} \rangle_{\psi}$ como
 
\begin{align*}
\langle \Tr (\omega)^{2} \rangle_{\psi} &= \langle \Tr(\omega \otimes \omega)F) \rangle_{\psi}\\
&= \Tr(\$ \otimes \$ [ \langle \ket{\psi} \bra{\psi} \otimes \ket{\psi} \bra{\psi} \rangle_{\psi}]F)\\
&=\Tr \Bigg( \$ \otimes \$ \Bigg[ \frac{\Pi_{RR}(\mathbbm{1}+F)}{d_{R}(d_{R}+1)} \Bigg]F \Bigg)\\
&= \sum_{kl} \Tr \Bigg(  \ket{kl}\bra{kl}  \Bigg(  \frac{\Pi_{RR}(\mathbbm{1}+F)}{d_{R}(d_{R}+1)} \Bigg) \ket{kl}\bra{kl} F \Bigg)\\
&=\sum_{kl} \Tr ( \ket{kl}\bra{lk} ) \Bigg( \frac{\bra{kl} \Pi_{RR}(\ket{kl}+\ket{lk})}{d_{R}(d_{R}+1)}  \Bigg)\\
&=\sum_{k} \frac{2 \bra{kk} \Pi_{RR}\ket{kk}}{d_{R}(d_{R}+1)} \\
&\leq \sum_{k} \frac{2 \bra{k} \Pi_{R}\ket{k}}{d_{R}(d_{R}+1)} < \frac{2}{d_{R}}
\end{align*}
se sigue directamente que 
\begin{equation}
\langle d^{eff}(\omega) \rangle_{\psi} = \Bigg \langle \frac{1}{\Tr (\omega)^{2}} \Bigg \rangle_{\psi} \geq \frac{1}{\langle \Tr (\omega)^{2} \rangle_{\psi}} > \frac{d_{R}}{2}
\end{equation}
concluyendo la prueba.
\item
Para demostrar la segunda parte se usará el lema de levy \cite{lema} pero no directamente sino sobre la función 
\begin{equation}
f(\psi) \equiv f(\vec{x}(\psi)) = \ln \Bigg( \Tr \Big( \tilde{\$}[ \ket{\psi}\bra{\psi}]^{2} \Big) \Bigg)
\end{equation}

Donde el operador  $ \tilde{\$} $ actúa sobre el subespacio $\mathcal{H}_{T} \subseteq \mathcal{H}$ generado por los estados de energía con proyección diferente de cero sobre $\mathcal{H}_{R}$ (estados que satisfagan $\bra{k} \Pi_{R} \ket{k} \neq 0$). El subespacio $\mathcal{H}_{T}$ contiene todos los estados que pueden aparecer durante la evolución temporal del estado inicial en $\mathcal{H}_{R}$ , y $\ \widetilde{\$} $ mapea estos estados de regreso a $\mathcal{H}_{R}$ de acuerdo a 
\begin{equation}
 \tilde{\$}[\rho]= \sum_{k}  \tilde{\ket{k}} \bra{k} \rho \ket{k} \tilde{\bra{k}} \quad y \quad  \tilde{\ket{k}}= \frac{1}{\sqrt{\bra{k}\Pi_{R}\ket{k}}} \Pi_{R} \ket{k}.
\end{equation}
nótese que cuando el Hamiltoniano conmuta con $\Pi_{R}$, $ \tilde{\$}$ es idéntico al $ \$ $ en $\mathcal{H}_{T}$. Calculando el promedio de la función se encuentra
\begin{align*}
\Big \langle \ln \Big( \Tr \Big( \tilde{\$}[ \ket{\psi}\bra{\psi}]^{2} \Big) \Big) \Big \rangle_{\psi} & \leq \ln \Big \langle \Tr \Big( \tilde{\$}[\ket{\psi}\bra{\psi}]^{2} \Big) \Big \rangle_{\psi}\\
&= \ln \Tr\Big( \tilde{\$}\otimes \tilde{\$} [\langle \ket{\psi}\bra{\psi} \otimes \ket{\psi}\bra{\psi} \rangle_{\psi}] F \Big)\\
&= \ln \bigg( \Tr \bigg( \tilde{\$}\otimes \tilde{\$} \bigg[ \frac{\Pi_{RR}(\mathbbm{1}+F)}{d_{R}(d_{R}+1)} \bigg] F   \bigg) \bigg)\\
&= \ln \Bigg( \sum_{kl} \Tr \bigg( \tilde{\ket{kl}} \bra{kl} \bigg( \frac{\Pi_{RR}(\mathbbm{1}+F)}{d_{R}(d_{R}+1)} \bigg) \ket{kl}  \tilde{\bra{lk}}  \bigg) \Bigg)\\
&= \ln \Bigg( \sum_{kl} \tilde{\bra{lk}}\tilde{\ket{kl}} \bigg( \frac{\bra{kl}\Pi_{RR}(\ket{kl}+\ket{lk})}{d_{R}(d_{R}+1)} \bigg) \Bigg)\\
&\leq \ln \bigg( \frac{2}{d_{R}(d_{R}+1)} \sum_{kl} \bra{lk}\Pi_{RR}\ket{kl}  \bigg)\\
&= \ln \bigg(  \frac{2}{d_{R}(d_{R}+1)} \sum_{k} \bra{k}\Pi_{R} \ket{k}  \bigg) < \ln (\frac{2}{d_{R}}).
\end{align*}
Para acotar la constante de Lipchitz  de la función $f(\psi)$ se usará otra función
\begin{equation}
g(\psi)= \ln \Tr \Bigg[  \bigg( \sum_{n} \ket{\hat{n}}\bra{\hat{n}} \tilde{\$} [\ket{\psi}\bra{\psi}] \ket{\hat{n}}\bra{\hat{n}} \bigg)^{2}  \Bigg]
\end{equation}
donde $\ket{\hat{n}}$ es una base ortonormal de $\mathcal{H}_{R}$. Escribiendo

\begin{equation}
t_{nk0}= \Re [ \bra{\hat{n}} \tilde{\ket{k}} \bra{k}\ket{\psi}] \quad y \quad  t_{nk1}= \Im [ \bra{\hat{n}} \tilde{\ket{k}} \bra{k}\ket{\psi}]
\end{equation}
se sigue que 
\begin{equation}
g(\psi)= \ln \Tr \Bigg[  \Bigg( \sum_{nkz} t_{nkz}^{2}  \ket{\hat{n}} \bra{\hat{n}}  \Bigg)^{2} \Bigg]= \ln \sum_{n} \Bigg( \sum_{kz} t_{nkz}^{2} \Bigg)^{2}.
\end{equation}

Para encontrar la constante de Lipschitz de $g$ es suficiente con encontrar una cota superior al gradiente 
\begin{equation}
\frac{\partial g}{\partial t_{nkz}}=\frac{1}{\sum_{n'} (\sum_{k'z'}t_{n'k'z'}^{2})^{2}} 2.2.t_{nkz} \sum_{k'z'}t_{nk'z'}^{2}.
\end{equation}
Introduciendo la notación $p_{n}=\sum_{kz}t_{nkz}^{2}$, y notando que $\sum_{n} p_{n}=1$, se encuentra 
\begin{align*}
|\nabla g|^{2} &= \sum_{nkz} \bigg(\frac{\partial g}{\partial t_{nkz}} \bigg )^{2} \\
&= \frac{16 \sum_{n} p_{n}^{3}}{(\sum_{n} p_{n}^{2})^{2}}\\
&\leq \frac{16(\sum_{n} p_{n}^{2})^{3/2}}{(\sum_{n} p_{n}^{2})^{2}}\\
&=\frac{16}{(\sum_{n} p_{n}^{2})^{1/2}}\\
&\leq 16 \sqrt{d_{R}},\\
\end{align*}
Por lo tanto la constante de Lipschitzs de $g$ llega hasta $4 \sqrt[4]{d_{R}}$. Para obtener la constante de Lipchistz de $f$ se nota que $g(\psi) \geq f(\psi)$ la igualdad se da si $ \{ \ket{ \hat{n}} \} $ es una base  propia de $\Tr_{B} \ket{\psi}\bra{\psi}$. Ahora para dos vectores cuales quiera, sin perdida de generalidad asumir $f(\psi_{1}) \leq f(\psi_{2})$, y tomar $ \{ \ket{\hat{n}} \}$ como la base propia de $\Tr_{B} \ket{\psi}\bra{\psi}$. Entonces,
\begin{equation}
f(\psi_{1})- f(\psi_{2})\leq g(\psi_{1})- g(\psi_{2}) \leq 4\sqrt[4]{d_{R}} |\ket{\psi_{1}}-\ket{\psi_{2}}|_{2},
\end{equation}
entonces la constante de Lipschitz para $f$ está acotada por $4\sqrt[4]{d_{R}}$.
Aplicando el lema de Levy a $f(\psi)$ y  observando que $\Pr \{ x>a \} \leq b$ y $x \geq y$ implica $\Pr \{ y>a \} \leq b$  sustituyendo la cota en $\langle f(\psi) \rangle_{\psi}$ obtenida arriba entonces 

\begin{equation}
\ln\Big( \Tr (\tilde{\$}[\ket{\psi}\bra{\psi}]^{2}) \Big) \geq \ln ( \Tr ( \$ [\ket{\psi}\bra{\psi}]^{2}) )
\end{equation}

esto da 

\begin{equation}
\Pr_{\psi}  \bigg \{ \ln (\Tr ( \$ [\ket{\psi}\bra{\psi}]^{2})) > \ln\frac{2e^{\epsilon}}{d_{R}} \bigg \} \leq 2 \exp (-\frac{\epsilon^{2} \sqrt{d_{R}}}{72 \pi^{3}})
\end{equation}

Tomando la desigualdad que está dentro de los corchetes  multiplicándola por menos y tomando su exponencial se llega

\begin{equation}
\Pr_{\psi} \bigg \{  d^{eff}(\omega) < \frac{d_{R}}{2e^{\epsilon}} \bigg \} \leq 2\exp(-\frac{\epsilon^{2}\sqrt{d_{R}}}{72\pi^{3}}),
\end{equation}

y haciendo que $\epsilon=\ln 2$ se llega al resultado esperado.
\\
\end{enumerate}

La primera parte del teorema \ref{teorema2} nos habla de cómo el promedio de la dimensión efectiva es más grande que la dimensión del subespacio de Hilbert esto significa que si se tiene estados de un subespacio bastante grande se puede asegurar un $d^{eff}(\omega)$ grade esto implica un $\langle D(\rho_{S}(t) ,\omega_{S}) \rangle_{t}$ pequeño. La segunda parte solo confirma de manera más estricta el hecho de encontrar un $d^{eff}(\omega)$ pequeño, mostrando que la probabilidad de encontrar una dimensión efectiva menor a $\frac{d_{R}}{4}$ es exponencialmente pequeña.\\
\\
Usando en teorema \ref{teorema2} se puede ver qué ocurre con un estado escogido de forma aleatoria del espacio total $\mathcal{H}$, un estado genérico. El análisis se sigue simplemente poniendo $d_{R}=d$ gracias a esto se comprende que $d^{eff}(\omega) \sim d$ ya que  hay una probabilidad exponencialmente baja para que se dé  el caso en que $d^{eff} < \frac{d}{4}$, como $d=d_{S}d_{B}$ la cota para las fluctuaciones queda $\sqrt{\frac{d_{S}}{d_{B}}}$. Un sistema de muchas partículas la dimensión del espacio de Hilbert crece de manera exponencial \cite{TodaStat}, si el subsistema es una fracción constante del número de partículas del baño esta proporción caerá de manera exponencial con el número total de de partículas luego los subsistemas se equilibrarán.
\\
Se podría suponer que  con los teoremas presentados hasta el momento el problema de termalización se ha resulto mayoritariamente pero qué ocurre con los sistemas que están lejos del equilibrio. Con lo dicho arriba sobre los estados genéricos se pensaría que cualquier sistema debe llegar al equilibrio pero esto no es cierto debido a que los estados lejos del equilibrio no son genéricos; los estados lejos del equilibrio no son típicos, por el capítulo anterior se sabe que con cotas exponenciales la mayoría de los estados en el espacio de Hilbert son tales que un subsistema pequeño está en un estado canónico.
\\
Para poder sacar algo de esta pregunta se planteará la situación común, hay un baño que consiste de un número muy grande de partículas de las cuales se conoce unos parámetros macroscópicos, dentro de este se pone un subsistema con un estado inicial arbitrario pero descorrelacionado con el ambiente. Ahora la pregunta es ¿ el subsistema se equilibra?, se verá que para cualquier estado inicial del subsistema y para casi todos los estados iniciales del baño el subsistema se equilibra. Esto incluye cuando el subsistema está lejos del equilibrio.
\\
El estado inicial del sistema está dado por $\ket{\Psi}_{SB}=\ket{\psi}_{S} \ket{\psi}_{B}$. El estado del subsistema es uno arbitrario $\ket{\psi}_{S}$ en el espacio de Hilbert. Dado unos parámetros macroscópicos el baño  debe cumplir con estos, luego el estado del baño $\ket{\phi}_{B} \in \mathcal{H}_{B}^{R} \subseteq \mathcal{H}_{B}$. Esta restricción mantiene la generalidad de seguir en cualquier espacio de Hilbert restringido. Pero la restricción es solo inicial al evolucionar el baño en el tiempo este puede moverse fuera de $\mathcal{H}_{B}^{R}$.
\\
Usando el teorema \ref{teorema2} para $\mathcal{H}_{R}= \ket{\psi}_{S} \otimes \mathcal{H}_{B}^{R}$ entonces $d_{R}=d_{B}^{R}$ esto da como resultado que para casi todos los estados iniciales del baño y cualquier estado del subsistema este se equilibrará para estas condiciones, mientras que $d_{B}^{R} >> d_{S}^{2}$. El mecanismo en que el subsistema se equilibra puede ser bastante complicado ya que el baño pasa por muchos estados diferentes y no llega el equilibrio. Aunque el baño no llegue al equilibrio y se salga del subespacio $\mathcal{H}_{B}^{R}$ el subsistema  puede equilibrarse de todas formas. 
\\
En principio puede que la evolución del subsistema sea sensible a la forma precisa del baño. Para ver que el baño no se equilibra de manera genérica se verá que $d^{eff}(\omega_{S})$ es mucho mayor que $d^{eff}(\rho_{B}(t))$ lo cual muestra que el baño sigue evolucionando y no se equilibra en ningún estado. Como los dos sistemas están en un estado puro $\rank(\rho_{B}(t))=\rank(\rho_{S}(t)) \geq d_{S}$ como la dimensión efectiva de un estado es siempre menor a su rango se obtiene 

\begin{equation}
d^{eff}(\rho_{B}(t)) \leq d_{S}
\end{equation}

pero 

\begin{equation}
d^{eff}(\omega_{B}) \geq \frac{d_{eff}(\omega)}{d_{S}}
\end{equation}

Pero para un estado genérico la segunda parte del teorema \ref{teorema2}  dice $d^{eff}(\omega) > \frac{d_{R}}{4}$ se tiene

\begin{equation}
d^{eff}(\omega_{B}) \geq \frac{d^{R}}{4d_{S}}=\frac{d_{B}^{R}}{4d_{S}} \gg d_{S} \geq d^{eff}(\rho_{B}(t)).
\end{equation}


Lo trabajado anteriormente muestra cómo susbsistemas de dimensión pequeña en comparación con el ambiente se equilibrarán. Ahora se verá cual sería la dependencia del estado de equilibrio del subsistema. Hasta ahora el estado inicial podría hacer que el equilibrio sea un estado diferente dependiendo del inicial. Sea el estado de equilibrio del subsistema $\omega_{S}^{\psi}$.Como es conocido se desearía que el estado de equilibrio dependa solo de los parámetros macroscópicos y no del estado inicial microscópico.
\\
El teorema siguiente prueba que para casi todos los estados en un subsistema restringido llevan al mismo estado de equilibrio.
\begin{theorem} \label{mismo estado}
\begin{enumerate}
\item Casi todos los estados iniciales de un subespacio restringido bastante grande llevan al mismo estado de equilibrio de un subsistema pequeño. En particula, con $\langle .\rangle_{\psi}$ siendo el promedio sobre estados puros aleatoriamente uniformes $\ket{\psi(0)} \in \mathcal{H}_{R} \subset \mathcal{H}_{S} \otimes \mathcal{H}_{B}$ y $\Omega_{S}= \langle \omega_{S}^{\psi} \rangle_{\psi}$ :
\begin{equation}
\langle D(\omega_{S}^{\psi}, \Omega_{S}) \rangle_{\psi} \leq \sqrt{\frac{d_{S} \delta}{4d_{R}}} \leq \sqrt{\frac{d_{S}}{4d_{R}}}.
\end{equation}
La primera desigualdad es más estricta pero más complicada 
\begin{equation}
\delta= \sum_{k}\bra{E_{k}}\frac{\Pi_{R}}{d_{R}}\ket{E_{k}} \Tr_{S} ( \Tr_{B} (\ket{E_{k}} \bra{E_{k}}) )^{2} \leq  1,
\end{equation}
donde $\Pi_{k}$ es el proyector sobre $\mathcal{H}_{R}$.
\item Para un estado aleatorio $\ket{\psi} \in \mathcal{H}_{R}\subset \mathcal{H}$, la probabilidad que $D(\omega_{S}^{\psi},\Omega_{S}) > \frac{1}{2} \sqrt{\frac{d_{S} \delta}{d_{R}}}+ \epsilon$ caiga exponencialmente con $\epsilon^{2}d_{R}$:
\begin{equation}
\Pr_{\psi} \bigg \{ D(\omega_{S}^{\psi}, \Omega_{S}) > \frac{1}{2}\sqrt{\frac{d_{S} \delta}{d_{R}}} + \epsilon \bigg \} \leq 2exp(-C'\epsilon^{2}d_{R}),
\end{equation}
con $C'=\frac{2}{9 \pi^{3}}$. Si se pone $\epsilon=d_{R}^{-1/3}$ da una distancia promedio pequeña con alta probabilidad cuando $d_{R}>>d_{S}$.
\end{enumerate}
\end{theorem}
\textbf{Demostración}
\begin{enumerate}
\item Se usa la relación ya establecida entre la distancia de traza y la distancia de Hilbert-Schmidt
\begin{align*}\label{a12}
\langle D(\omega_{S}, \langle \omega_{S} \rangle_{\psi}) \rangle_{\psi} &\leq \bigg \langle \frac{1}{2} \sqrt{d_{S} \Tr (\omega_{S}, \langle \omega_{S} \rangle_{\psi})^{2}} \bigg \rangle_{\psi}\\
&\leq \frac{1}{2} \sqrt{d_{S} \Big \langle \Tr (\omega_{S} - \langle \omega_{S} \rangle_{\psi})^{2} \Big \rangle_{\psi}}.
\end{align*}
Ahora se miran las cotas del término que es promediado
\begin{align*}
\Big \langle \Tr (\omega_{S}, \langle \omega_{S} \Big \rangle_{\psi})^{2} \rangle_{\psi}  &= \big \langle \Tr(\omega_{S}^{2})_{\psi} \big \rangle - \Tr_{S} (\langle \omega_{S}^{2} \rangle_{\psi})\\
&=\Tr (   (\langle \omega_{S} \otimes \omega_{S} \rangle_{\psi} - \langle \omega_{S} \rangle_{\psi} \otimes \langle \omega_{S} \rangle_{\psi} )F )\\
&= \Tr_{SS} \Big( \Tr_{BB} \big( \$ \otimes \$ \big[ \langle \ket{\psi}\bra{\psi} \otimes  \ket{\psi}\bra{\psi}\rangle_{\psi} - \frac{\Pi_{R}}{d_{R}} \otimes  \frac{\Pi_{R}}{d_{R}}  \big]  \big) F \Big)\\
&= \Tr_{SS} \bigg( \Tr_{BB} \bigg( \$ \otimes \$ \bigg[ \frac{\Pi_{RR}(\mathbbm{1}+F)}{d_{R}(d_{R}+1)} - \frac{\Pi_{RR}}{d_{R}^{2}}  \bigg]  \bigg) F \bigg)
&\leq \Tr_{SS} \bigg(  \Tr_{BB} \bigg( \$ \otimes \$ \bigg[ \frac{\Pi_{RR}F}{d_{R}^{2}} \bigg] \bigg) F \bigg)\\
&= \sum_{kl}  \Tr_{SS} \Big(  \Tr_{BB} \Big( \ket{kl}\bra{kl} \frac{\Pi_{RR}}{d_{R}^{2}}  \ket{lk}\bra{kl} \Big) F \Big)\\
&= \sum_{kl} \frac{\bra{kl}\Pi_{RR}\ket{lk}}{d_{R}^{2}} \Tr_{S}(\Tr_{B}(\ket{k}\bra{k}) \Tr_{B}(\ket{l}\bra{l}) )\\
&\leq \sum_{kl} \frac{\bra{k}\Pi_{R}\ket{l}\bra{l}\Pi_{R}\ket{k}}{d_{R}^{2}} \Tr_{S} \Big( \frac{(\Tr_{B}(\ket{k}\bra{k}))^{2}+(\Tr_{B}(\ket{l}\bra{l}))^{2}}{2} \Big)\\
&=\frac{1}{d_{R}} \sum_{k} \bra{k} \frac{\Pi_{R}}{d_{R}} \ket{k} \Tr_{S} \Big( (\Tr_{B} \ket{k}\bra{k})^{2} \Big)\\
&\leq \frac{1}{d_{R}} \sum_{k} \bra{k} \frac{\Pi_{R}}{d_{R}} \ket{k} = \frac{1}{d_{R}}
\end{align*}
en la segunda desigualdad se usó el hecho que $\Tr_{S} (\Tr_{B} \ket{k}\bra{k}-\Tr_{B} \ket{l}\bra{l})^{2}\geq 0$ es las traza de un operador positivo. Ahora insertando la tercera línea contando desde el final y con la relación entre la distancia de traza y la distancia de Hilbert-Schmidt se demuestra el resultado.
\item Para demostrar la segunda parte que todos los estados $\ket{\psi}$ llevan al mismo estado de equilibrio, se usa el ya conocido lema de Levy. Se aplica este lema directamente a la función $f(\psi) \equiv D(\omega_{S}^{\psi}, \Omega_{S})$ en la hiperesfera de dimensión $2d_{R}-1$ de estados cuánticos. Para hallar la constante de Lipschitz se sigue de la misma manera que en el capítulo anterior
\begin{align*}
|D(\omega_{S}^{\psi_{1}}, \Omega_{S})-D(\omega_{S}^{\psi_{2}},\Omega_{S})| &\leq D(\omega_{S}^{\psi_{1}},\omega_{S}^{\psi_{2}})
\\
&\leq D(\omega^{\psi_{1}},\omega^{\psi_{2}})
\\
&= \sqrt{1- |\bra{\psi_{1}} \ket{\psi_{2}}|^{2}}
\\
&\leq |\ket{\psi_{1}} - \ket{\psi_{2}}|_{2}.
\end{align*} 
entonces la constante de Lipschitz satisface $\eta \leq 1$. Sustituyendo en el lema de Levy junto con el valor promedio encontrado en la primera parte de esta demostración se llega al resultado deseado.
\\
\end{enumerate}

Para un estado inicial que sea el producto del estado del sistema y el del ambiente $\ket{\psi}_{SB}= \ket{\psi}_{S} \ket{\phi}_{B}$ en el espacio $\mathcal{H}_{R}=\ket{\psi} \otimes \mathcal{H}_{R}^{B}$ se mostró que para estados genéricos el ambiente causa que el subsistema se equilibre aunque el estado de equilibrio $\omega_{S}^{\psi}$ del subsistema podría  depender del estado inicial del baño $\ket{\phi}_{B}$. Esto no es así y casi todos los estados del baño en $\mathcal{H}_{B}^{R}$ llevan al mismo estado de equilibrio del subsistema simplemente usando el resultado anterior a  $\mathcal{H}_{B} = \ket{\psi}\otimes \mathcal{H}_{R}^{B}$ y luego $d_{R} = d_{R}^{B}$, si $d_{R}^{B} \gg d_{S}$ para casi todos los estados iniciales del baño llevarán al subsistema al mismo estado de equilibrio $\omega_{S}$. como se usó la cota menos estricta este resultado no depende de la forma explícita de los estados de energías propios.
\\


El nuevo enfoque dado en el capítulo anterior fue explorado aún más siguiendo a \cite{LindenPaper} en este capítulo viendo qué ocurre con los sistemas al evolucionar. El enfoque dado por \cite{Popescu2006} es fructífero  en resultados y se puede seguir profundizando esto genera un interés para que se siga explorando esta perspectiva. Este capítulo mostró cómo en general los estados de un espacio de Hilbert bastante grande que tiene muchos estados propios de energía llegan a estar bastante cerca del estado promedio (promediado temporalmente) y se dice que se equilibra este. Luego se muestra que en general el estado de equilibrio es ese estado promedio. Linden et al. mostraron que hay equilibrio para estados genéricos y también para estados fuera del equilibrio además mostraron la independencia del estado del baño. Aunque ellos no mostraron la independencia del estado del subsistema no debería influenciar el equilibrio. Siguiendo lo dicho por ellos este problema muestra más dificultades que los otros dado a que es necesario una especificación de la energía porque ahora jugaría un papel más importante, en los teoremas anteriores no fue necesario hablar de la energía excepto en el teorema \ref{mismo estado} donde hay una desigualdad dada por los estados propios de energía.
%-----------------------------
%\textbf{Independenciadelestado del sistema}: Como yase hadicho el subsistema en contacto con el baño llega al equilibrio sin importar su estado inicial pero sí depende del baño. Este resultado no se pudo resolver pero se da más indicios.
%Los problemas son que el estado de equilibrio no siempre es independiente del estado inicial del subsiste,a. si el subsistema puede cambiar al ambiente de forma drástica el estado de equilibrio sí dependerá del estado inicial del subsistema. aquí muestra que las dimensiones del espacio de Hilbert del subsistemas el ambiente son importante pero el valor se le haya dado una restricción al Hamiltoniano (Brechas de energías no degeneradas ) esta condición no están estricta como para no permitir cantidades conservadas en el subsistema cuando estas cantidades conservadas existen los estados del subsistema con diferentes constantes(estados inicial es diferentes )no podrán llegar al mismo estado de equilibrio.
%-----------------

