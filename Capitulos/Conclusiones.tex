\chapter{Conclusiones}

En este escrito se hizo un recuento histórico de la termodinámica para llegar a la mecánica estadística. Desde esta perspectiva histórica se mostró la evolución de los distintos conceptos que han fundamentado la mecánica estadística. Se expuso la perspectiva de Boltzmann junto con las dificultades que ésta acarrea, como las objeciones dadas a su función H y la relación con la entropía termodinámica. A su vez, se exhibió la perspectiva de Gibbs, la cual se condensa en la idea del ensamble. También se señalaron los inconvenientes que su propuesta exhibe, como el concepto de copias imaginarias. \\
Junto con esto se muestra que hay otras perspectivas más modernas como la de Jaynes y su máxima entropía, que fue la que se tuvo en cuenta en el desarrollo de este texto. Esta teoría forma el primer vínculo entre la mecánica estadística y la teoría de la información. La perspectiva de Jaynes mejora respecto a las anteriores porque no introduce hipótesis a priori y especifica que la probabilidad es vista desde un sentido subjetivo, algo que las otras perspectivas no consideraban.
\\
Más adelante se siguió el artículo de Popescu et al  \cite{Popescu2006} para mostrar la tipicidad canónica con ayuda de las herramientas de la teoría de la información cuántica. Se replicaron los cálculos del artículo para explicar los detalles del formalismo que los autores propusieron. Con este enfoque se vio una manera en la que era posible conciliar las perspectivas de Gibbs y Boltzmann. Además, el enfoque muestra que el principio de probabilidades a priori se puede reemplazar por el principio general canónico. De esto se concluye que es muy poco probable encontrar un estado que no se encuentre en el estado canónico.
\\
Para terminar, se vio las extensiones que tienen los nuevos fundamentos gracias a la tipicidad canónica. Se vio la capacidad del principio general canónico para explicar cómo se termaliza un sistema. Se reprodujo el análisis hecho por Linden et al \cite{LindenPaper} en este tema. Aunque no pudieron demostrar la independencia del subsistema, abrieron campo a nuevas investigaciones.