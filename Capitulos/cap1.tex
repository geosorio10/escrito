\chapter{El entrelazamiento y la mecánica estadística}

La idea principal que se quiere mostrar es cómo se puede reemplazar el postulado de probabilidades iguales por un principio canónico general basado en el entrelazamiento cuántico, basado en el entrelazamiento del sistema y el ambiente.

La mecánica clásica nos habla de un sistema físico definido que para todos los tiempos está especificado. Este sistema evolucionan de manera determinista. Pero lo que sorprende al tratar sistemas termodinámicos es que aunque se hable de un sistema clásico este puede mostrar propiedades que dependan de promedios estadísticos. La conexión que hay entre el determinismo y estas probabilidades es una discusión que lleva desde los inicios de la termodinámica. Aunque el planteamiento que se mostrará no necesita de los métodos típicos de la mecánica estadística como por ejemplo: aleatoriedad subjetiva, promedio sobre el ensamble o promedio temporal; Esto le da una fuerza a estas ideas ya que no debe entrar en problemas de ergodicidad. 

Este nuevo enfoque se tiene un universo que está compuesto por el sistema y el ambiente. Dando como condición que el ambiente sea los suficientemente grande. Este universo está descrito por un estado cuántico puro ( se conoce el estado de manera exacta) que obedece una restricción global, que el sistema alcance el equilibrio térmico por medio de la interacción mutua (termalización) es un producto del entrelazamiento del sistema y el ambiente (Popescu et al,2006). Más adelante se le dará una definición más rigurosa que ayudará a dar cotas para la expresión "ambiente suficientemente grande". Con este enfoque se quiere formular un principio canónico general  el sistema estará termalizado para casi todos los estados puros del universo, dando límites cuantitativos. La restricción que se impone no es una específica esto generaliza los resultados tradicionales dados en la literatura donde se toma por restricción la energía (cualquier libro, año).
Estos resultados no miran la evolución del sistema pero debido a que la mayoría de los estados del universo están termalizados se preve que no importa el estado inicial cualquier evolución llevará al estado a uno en el equilibrio.

(
Lo que mi asesor no me ha respondido.
)

Ya poniendo las ideas más explícitas se supone tener un sistema cuántio aislado y grande que se llamará el universo, que se descompone en dos partes el sistema $S$ y el ambiente $E$. La dimensión del ambiente es mucho maś grande que la del sistema $S$. Además se le impone una restricción global al universo llamada $R$. Desde la mecánica cuántica estas condiciones se pueden poner como restricciones en el espacio de Hilbert, restricción de los estados posibles:

\begin{equation}
\mathcal{H_{R}}\subseteq \mathcal{H_{S}}\otimes \mathcal{H_{E}},
\end{equation}

Donde $\mathcal{H_{S}}$ y $\mathcal{H_{E}}$ son los espacios de Hilbert del sistema y el ambiente con dimension $d_{\mathcal{S}}$  y $d_{\mathcal{E}}$ respectivamente. Es bueno recalcar que $R$ es una restricción arbitraria generalmente se toma como la energía del universo.
Ahora se define el estado equiprobable del universo bajo $R$ como

\begin{equation}
\mathcal{E_{R}} = \frac{1}{d_{R}} \mathbbm{1_{\mathcal{R}}},
\end{equation}

Donde $\mathbbm{1_{\mathcal{R}}}$ es el operador identidad (proyección) sobre el espacio de Hilbert  $\mathcal{H_{R}}$  que tiene dimensión $d_{\mathcal{R}}$. Esto se relaciona con el principio de probabilidades iguales porque este es el estado máximamente mezclado en $\mathcal{H_{R}}$ por ser así todos los estados bajo la restricción de R tienen la misma probabilidad de salir.

Definimos $\Omega_{\mathcal{S}}$ como el estado canónico que está restringido por $R$ cuando el universo se encuentra en el estado $\mathcal{E_{R}}$.Esto significa que si se hace una traza parcial al universo sobre el ambiente da como resultado el estado canónico:
\begin{equation}
 \Omega_{\mathcal{S}} =Tr_{\mathcal{E}} \mathcal{E_{R}}.
\end{equation}

Aquí se hace un supuesto importante y es que el universo está en un estado puro $\ket{\phi}$ y no en un estado mixto $\mathcal{E_{R}}$, esto quiere decir que se conoce todo lo que es permitido por la mecánica cuántica del universo si estuviese en un estado mixto significaría que nosotros no tenemos toda la información que se pudiese tener. Ahora lo que se quiere ver es que aun que el estado del universo sea puro el estado reducido del sistema,

\begin{equation}
\rho_{S}=Tr_{E}\ket{\phi}\bra{\phi} 
\end{equation}

se acerca al estado canónico para la gran mayoría de los casos 

\begin{equation}
\rho_{S}=\Omega_{S}.
\end{equation}

O sea que para casi todos los estados puros $\ket{\phi} \in \mathcal{H_{R}}$ del universo el sistema se comporta como si el universo estuviese en el estado mixto equiprobable $\mathcal{E_{R}}$. Siendo este el principio general canónico.

Clarificando la idea, el estado canónico $\Omega_{S}$ del sistema es el estado del sistema cuando el universo se encuentra en el estado equiprobable $\mathcal{E_{R}}$. Se puede interpretar el principio general canónico como un principio que estipula que las probabilidades iguales del sistema son aparentes. Porque para casi cualquier estado del universo que sea puro, un subsistema de este universo que cumpla con ser lo suficientemente pequeño se comporta como si el universo estuviese en el estado equiprobable $\mathcal{E_{R}}$. Cabe recordar que aún no se ha especificado la restricción $R$ entonces todo este análisis es general, la restricción no necesariamente debe ser la energía u otras cantidades que se conserven. Esto hace que $\Omega_{S}$ no  deba ser obligatoriamente es el estado canónico usual, puede ser el gran canónico o cualquier otro que sea acorde con la restricción. Este principio puede ser de utilidad cuando la interacción entre el ambiente y el sistema no es débil como siempre se toma.


