\chapter{Introducción}

\section{Recuento Histórico}

La mecánica estadística es la base de la termodinámica pero históricamente primero apareció la termodinámica. La termodinámica fue la agrupación de varias ideas en el siglo diecinueve que encerraban varios tipos de experimentos  y conceptos que desde la antigua Grecia. Aunque las ideas que surgieron en la antigüedad se han ido refinando como ejemplo Klaudios Galenos (133-200 A.C) suponía que la influencia del clima sobre los fluidos del cuerpo podría determinar el carácter de una persona. Así decía que los habitantes del norte frío y húmedo eran salvajes y violentos, al contrario de los habitantes del sur, caliente y seco, eran flácidos y mansos. Esto muestra que la idea de temperatura ha sido conocida de alguna u otra manera. Ya en el año 1578  Johannis Hasler presentó una tabla de las temperaturas corporales de las personas en relación a la altura en la que vivían. Pero como se conoce ahora mientras se tenga una buena salud no importa la altura se debe tener la misma temperatura corporal. Este error fue debido a los instrumentos de medición usados en esa época. Ya en el inicio del siglo 17 ya se tenía el termómetro. Gracias a este instrumento muchas ideas equivocadas dadas por la subjetividad de los sentidos sobre el calor y el frío se fueron eliminando. Ya en los años 1700 empezaron a aparecer las escalas de temperatura, hubo incluso 18 escalas en algún tiempo. En 1848 William Thompson (Lord Kelvin)  introdujo la escala absoluta o la escala Kelvin. Esta cuenta desde el cero absoluto en adelante, el punto de ebullición del agua a  1 atm es 373.15 Kelvin.
\\
Otra importante parte de la termodinámica es la energía pero en sus inicios no se habla de eso; era calor o fuerza. Pero no se comprendía qué era el calor. La teoría calórica pretendía dar explicación al calor, entre varias propuesta se pueden ver las de Pierre Gassendi (1592-1655) donde proponía que el calor y el frío eran tipos de materia diferente. Consideraba a los átomos del frío como tetraedros y al entrar a un líquido se solidificaría. Antoine Laurent Lavoisier (1743-1794) consideró al calor como otro elemento, junto con la luz, y lo consideraba como un fluido que llamaba el calórico. Una de las primeras personas en cuestionar la teoría calórica fue Graf Von Rumford(1753-1814), observando unos cañones y cómo cambiaba el calor liberado dependiendo de si estaban afilados  o no. Concluyó que el calor debía ser el mismo que fue administrado debido al movimiento. Rumford siguió trabajando en su idea del equivalente mecánico del calor, con dos caballo moviendo un cabrestante nota que el calor del barril es igual "al de 9 velas grande", más adelante precisó su hallazgo. Rumford siguió haciendo experimentos sobre el calor, midió de manera meticulosa el peso del agua antes y después de ser congelada. Encontró que el peso no cambiaba pero aún así dio calor en el proceso. Entonce concluyó que si el calórico existía era imponderable. Pese a los experimentos realizados por Rumford la teoría calórica siguió siendo la favorita por 40 años.	
\\
Robert Julius Mayer (1814-1878) estudio medicina pero su entusiasmo hacia la física lo guió para hacer experimentos en este campo. La idea principal que tenía Mayer era que la energía se conservaba de forma general. Es decir cualquier fenómeno capaz de aportar energía debía tenerse en cuenta al momento de un experimento. Gracias a sus estudios por su cuenta sabía que la energía cinética, fuerza viva en sus  palabras, podía convertirse en calor. Experimentalmente llegó a que la caída de un peso a una altura  de 365 metros correspondía a calentar el mismo peso de agua de $0^{\circ}$ a $1^{\circ}$. No es muy exacto pero estuvo bastante cerca, incluso llegó a cambiar la altura a 425 metros luego de que Joule hiciera mejores mediciones. Mayer tenía ideas bastante originales aunque en general no sabía expresarlas, por falta de experiencia matemática y su aislamiento de la comunidad científica. Pero Mayer fue el primero en hablar sobre máquinas térmicas y decir que el calor absorbido por el vapor era siempre mayor que el calor liberado durante la condensación y su diferencia era trabajo útil. Esta idea se expuso antes que Carnot y Clapeyron. Aunque tenía un manejo sobre el calor, no sabía la naturaleza verdadera del calor.
\\
James Prescott Joule (1818-1889) proporcionó, en extremo detalle, observaciones sobre el calor y la temperatura. Sus experimentos eran discutidos en sus escritos de forma muy extensa, dando presumibles errores, compensando perdidas, sus experimentos eran tan detallados que algunos de sus artículos se volvieron estándares. Es por eso que Mayer revisó sus experimentos respecto a los de Joule. Uno de los aportes de Joule fueron que gracias a sus experimentos se podía aceptar la conservación de la energía. Pero fue Hermann Ludwig Ferdinand Helmholtz(1821-1894) quien propuso la idea de que lo llamado calor era la energía cinética del movimiento térmico de los átomos. Fue Helmhotlz el primero en dejar la idea de un fuerza viva y llamarlo energía. EL trabajo de Helmholtz fue importante ya que Joule y  Mayer no podía dar una visión clara sobre lo que se llama calor, y divagaban entre calor y fuerza; Helmholtz dejó los conceptos más claros en comparación con ellos dos.  Todo este proceso hitórico es uno de los procesos para poder dar la primera ley de la termodinámica, se puede ver que la historia de la energía no es un proceso tan claro y llegar hasta la idea actual no fue para nada rápido ni sencillo. Puede ponerse en contraposición la mecánica cuántica que aunque fue inspirada también de la radiación del cuerpo negro, un tema que en ese entonces era tratado por la termodinámica, se construyó en menos años; aunque con ideas confusas pero su construcción matemática y teórica no fue demorada por siglos. Pero a diferencia de la energía, la entropía era un concepto más provocador al momento de darle un sentido microscópico. La historia de la  entropía empieza por los motores térmicos hay especulaciones que en el primer siglo antes de cristo ya existían algunas máquinas que funcionaban con vapor de agua, Hero de Alexandría se especula que hizo una de esa máquinas. Pero en el siglo 18 Thomas Newcome y Thomas Savery crearon una máquina de vapor que inicialmente ayudaba a sacar agua de las minas. Pero James Watt(1736-1819) mejoró esa máquina de vapor haciendo una que era de 3 a 4 veces más eficiente. Nicolas Léonard Sadi Carnot (1796-1832) se preguntó sobre que tanto se puede mejorar una máquina térmica. Carnot publicó un libro donde muestra aseveraciones que intentan resolver la pregunta hechas; una máquina que trabaja entre dos temperaturas y solo intercambia calor entre ellas, su eficiencia solo depende de la  las temperatura, $e=F(T)dT$ para un delta de temperatura, incluso aunque las máquinas se manejen con agentes diferentes para generar trabajo. Aunque Carnot seguía creyendo en la teoría calórica para llegar a estas ideas no fue necesario saber que esta teoría estaba incorrecta para llegar a sus resultados. Carnot no pudo encontrar exactamente los valores de la eficiencia. Clapeyron y Kelvin no pudieron tampoco encontrar los valores de esta eficiencia ni por mediciones ni por cálculos. Ya en 1850 se estaba asegurando que la teoría calórica estaba mal formulada y se debía hacer algo; Clausius retocó algunas ideas de Carnot y pudo encontrar la eficiencia del ciclo de Carnot. Clausius le dio forma a la termodinámica actual, hasta los cursos actuales de termodinámica siguen un artículo de Clausius que habla de gases ideales y vapor húmedo. Clausius amplió sus investigaciones a ciclos no infinitesimales y a procesos no reversibles, él fue quien dio la idea de la entropía y sus propiedades. Con eso propuso la segunda ley de la termodinámica: El calor no puede pasar solo de un cuerpo frío a un cuerpo caliente. Clausius llamó $S=\frac{Q}{T}$ la entropía y vio que esto era lo que se conservaba en un ciclo de Carnot. Pero en términos de los átomos qué significa $S$.
\\\
\section{teoría cinética}
Paralelamente a mitad del siglo 19 se trabajaba en la teoría cinética de los gases. Esta reposaba en dos suposiciones la primera que los gases eras sistemas mecánicos de muchas partículas idénticas. La segunda era que debido a la gran cantidad de partículas se debía introducir probabilidades para así ver algunas regularidades que salían debido a las configuraciones de las moléculas. Personas como Calusius, Maxwell y Boltzmann trabajaron en esta área y produjeron resultados importantes aunque algunos concordaban con los resultados experimentales, hubo muchas discusiones sobre las hipótesis usadas para llegar a estos resultados. Por ejemplo en los trabajos de Clausius utiliza hace uso de los siguientes supuestos para gases en reposo y en equilibrio térmico: Las moléculas que están dentro de un recipiente se encuentran distribuidas con la misma densidad en todo el recipiente, la distribución de velocidades es igual en todo el recipiente, todas las direcciones de velocidad son igual de probables. Estas tres hipótesis son el inicio para ver los sistemas mecánicos con una perspectiva probabilística. Maxwell convirtió las ideas de Clausius sobre la  poca dispersión de la distribución de velocidad en algo que se podía calcular. Así que Maxwell propuso en 1859 su ley de distribución
\begin{equation}
f(u,v,w) \Delta u \Delta v \Delta w = A e^{-B(u^{2}+v^{2}+w^{2})}  \Delta u \Delta v \Delta w.
\end{equation}
Donde $f(u,v,w) \Delta u \Delta v \Delta w $ es el número de moléculas entre esos límites de velocidades, cada límite representa un componente de la velocidad. Botlzmann en 1868 generalizó el resultado de Maxwell, ahora se tiene la misma situación de un gas en equilibrio y en reposo, pero ahora se tiene un campo de fuerza externo actuando sobre las moléculas. Además Boltzman tuvo en cuenta que las moléculas eran compuestas de otras partículas  lo cual afecta en los diferentes valores de energía potencial. Entonces denotando a $\Delta \tau$ como el rango de variaciones muy pequeñas que puede tener el estado de la molécula, o sea $\Delta \tau= \Delta x \Delta y \Delta z \Delta u \Delta v \Delta w$. La generalización dada por Boltzmann es
\begin{equation}
f(x,y,z,u,v,w) \Delta \tau = \alpha e^{-\beta E} \Delta \tau.
\end{equation}
$E$ es la energía total que tiene la molécula esto es la energía cinética, energía potencial interna y energía potencial externa. Esta generalización es llamada la distribución de Maxwell-Boltzmann. Boltzmann además mostró que cualquier otra distribución bajo la condición de colisiones entre partículas evoluciona hacía la distribución de Maxwell-Boltzmann, se usó el teorema H para concluir esto.

Para hablar del teorema H primero se debe explicar la ecuación de transporte de Boltzmann. La ecuación de transporte de Boltzmann puede ser derivada desde la jerarquía de Bogoliubov-Born-Green-Kirkwood-Yvon o mirando el problema del número de partículas en un rango del estado de una molécula en el espaico de fase. Para ambas rutas se debe hacer supuestos para poder seguir con el análisis, esto no se mostrará acá dado que no es pertinente seguir todos los pasos. La ecuación de transporte de Boltzmann describe cómo la distribución de moléculas $f(\vec{r},\vec{p},t)$, donde $\vec{r}$ es la posición y $\vec{p}$ es el momento,  evoluciona en el tiempo. La ecuación es 

\begin{align*}
(\frac{\partial}{\partial t}- \frac{\partial U}{\partial \vec{q}_{1}} \cdot \frac{\partial}{\partial \vec{p}_{1}} +\frac{\vec{p}_{1}}{m} \cdot \frac{\partial}{\partial \vec{q}_{1}})f 
\\
=- \int d^{3}\vec{p}_{2}d^{2} \Omega |\frac{d \sigma}{d \Omega}| |\vec{v}_{1}-\vec{v}_{2}|[f(\vec{p}_{1},\vec{q}_{1},t)f(\vec{p}_{2},\vec{q}_{1},t) - f(\vec{p}_{1}',\vec{q}_{1},t)f(\vec{p}_{2}',\vec{q}_{1},t)].
\end{align*}
Esta ecuación se puede leer como: el término a la izquierda describe el movimiento de una partícula en un potencial U, el término de la derecha es la probabilidad de encontrar una partícula con momento $\vec{p}_{1}$ en $\vec{q}_{1}$ ser alterada por una colisión con una partícula con momento $\vec{p}_{2}$.  La probabilidad de esta colisión está dada por el producto de los factores cinéticos proporcionado por la sección transversal diferencial $ |\frac{d \sigma}{d \Omega}|$, el flujo de partículas incidentes $|\vec{v}_{1}-\vec{v}_{2}|$ y la probabilidad de encontrar dos partículas $[f(\vec{p}_{1},\vec{q}_{1},t)f(\vec{p}_{2},\vec{q}_{1},t)$. El primer término substrae la probabilidad e integra sobre todos lo momentos posibles y el ángulo sólido. El segundo término es la adición del proceso inverso. 
El teorema H dice que si f satisface la ecuación de transporte de Boltzmann, entonces $\frac{dH}{dt} \leq 0$, donde 
\begin{equation}
H(t)= \int d^{3} \vec{p}_{1} d^{3} \vec{q}_{1}  f(\vec{p},\vec{q},t) \ln (f(\vec{p},\vec{q},t)).
\end{equation}
Con este teorema se puede demostrar que la distribución en el equilibrio es la distribución de Maxwell-Boltzmann. o sea cuando $t \to \infty$ la distribución  $f(\vec{p},t) \to f_{0}(\vec{p})=  \alpha e^{-\beta E}$. Pero hay un punto importante aquí y es algo que también en su época desconcertó a los contemporáneos de Boltzmann, ¿ cómo es posible llegar a una descripción de fenómenos irreversibles que muestran una dirección temporal pero se empezó con una teoría de un gas reversible?. Esto viene dado por las suposiciones echas al derivar la ecuación de transporte de Boltzmann. Se usa la hipótesis de caos molecular (Stosszahlanzatz), esta dice que después de una colisión las partículas están descorrelacionadas esto quiere decir $F(\vec{r},\vec{p}_{1},\vec{p}_{2},t)=f(\vec{r},\vec{p}_{1},t)f(\vec{r},\vec{p}_{2},t)$. Entonces es en este punto en el que la simetría temporal se rompe y hace que haya una dirección particular para el tiempo. 
El teorema H fue usado por Boltzmann para poder darle una base microscópica a la segunda ley. La relación es 
\begin{equation}
S(t)=- \frac{H(t)}{k_{B}}
\end{equation}
donde $S$ es la entropía y $k_{B}$ la constante de Boltzmann. Pero también Boltzmann recibió muchas críticas sobre  la posibilidad de asociar a H la entropía. Una de las objeciones que recibió fue sobre la posibilidad de poner todas las velocidades de las moléculas de forma opuesta a las dadas inicialmente, esto daría resultados opuestos a los que se esperaban. Si inicialmente se encontraba que H decrecía ahora al invertir las velocidades se halla que H aunmenta, esto significa que hay un caso para el cual la entropía disminuye lo cual va encontra de la segunda ley. Otro problema dado por E. Zermelo es que Poincaré había demostrado que en ese modelo cinético de un gas aislado, el sistema secomportaba quasiperiódico. Esto quiere decir que la función H puede ir asumiendo valores más grandes después de un periodo. 
\\
Ya más adelante W. Gibbs con su libro Elementary Principles of Statistica Mechanics en 1901 ya propuso la forma actual de hacer física estadística. El propuso las ideas que se expusieron en los preliminares, las ideas de ensambles microcanónico y canónico fueron gracias a él. Aunque Gibbs volviera la mecánica estadística más moderna no le quitaba algunos problemas; es más puso en ella otro problema con su idea de ensambles. Los promedios se encuentran gracias a los ensambles de varias copias idénticas del sistema pero en microestados diferentes que son admisibles por los parámetros macroscópicos. Pero no significa que el sistema que se está estudiando está en todos esos estados es solo un método para poder lograr describir una densidad en el espacio de fase, que luego se usa para poder calcular propiedades del sistema. Aquí entra la pregunta de que tan posible es hablar de copias imaginarias del sistema, sabiendo que solo existe uno. En el artículo hecho por E.T Jaynes se habla de las propuestas dadas por Boltzmann y Gibbs para la entropía. En este artículo pone las diferencias entre cada pero muestra que la entropía dada por Gibbs es la correcta. Siguiendo la notación del artículo,$d\Gamma \equiv d^{3}x_{1}...d^{3}p_{N}$, $d\Gamma_{1} \equiv d^{3}x_{1}d^{3}p_{1}$, $d\Gamma_{-1} \equiv d^{3}x_{2}...d^{3}p_{N}$; se define un ensamble por su función de distribución de $N$ partículas,$W_{N}(x_{1},x_{2},..,x_{N};p_{1},p_{2},...,p_{N};t)$, la cual da la densidad de probabilidad del sistame en todo  el espacio de fase. La función H de Gibbs esta dada por
\begin{equation}
H_{G}= \int W_{N} \log W_{N} d \Gamma,
\end{equation}
y la función H de Boltzman está dada por 
\begin{equation}
H_{B}=N \int w_{1} log w_{1} d\Gamma_{1}.
\end{equation}
Donde $w_{1}(x_{1},p_{1},t)$ es la probabilidad de densida de una sola partícula, 
\begin{equation}
w_{1}(x_{1},p_{1},t)= \int W_{N} d \Gamma_{-1}.
\end{equation}
En este artículo Jaynes muestra en el teorema 1 que 
\begin{equation}
H_{B} \leq H_{G},
\end{equation}
y la igualdad se cumple solo cuando $W_{N}(x_{1},x_{2},..,x_{N};p_{1},p_{2},...,p_{N})=w_{1}(x_{1},p_{1})..w_{1}(x_{N},p_{N},$,  es decir hay independencia entre las partículas esto es el caso para un gas que no tiene interacción entre partículas, gas ideal. Gracias a esto Jaynes prosigue con mostrar que la entropía de Boltzmann, $S_{B}=k_{B}H_{B}$, solo es cierta cuando se habla de un fluido con la misma densidad y temperatura en todo el espacio pero sin fuerzas entre partículas. Mientras la entropía de Gibbs, $S_{G}=k_{B}H_{G}$, es válida para cualquier sistema porque da la entropía fenomenológica de la termodinámica. Concluye que esta diferencia no puede ser pequeña porque hay interacciones entre partículas muy importantes que afectan ampliamente el resultado. Esto muestra que aunque la idea de Gibbs sobre un ensamble 





















\begin{comment}

En este corto capítulo se espera motivar la idea general de este escrito, mostrando un poco cuál ha sido el problema con los fundamentos de la mecánica estadística. En los siguientes capítulos se explicará el nuevo enfoque dado por Popescu et al \cite{Popescu2006}.
\\
La mecánica estadística estudia los cuerpos macroscópicos, los cuales están compuestos por un gran número de partículas. Los métodos que ayudan al análisis de estos cuerpos por lo general no dependen de la forma en que se pueda describir una sola partícula (ecuaciones de movimiento para una partícula) sino se respaldan en métodos probabilísticos. A inicios del siglo 19 la teoría de la termodinámica cogía fuerza y mostraba coherencia con los datos experimentales pero se quiso fundamentar la termodinámica desde una perspectiva microscópica. La entropía un concepto nacido en la termodinámica no tenía una clara idea de qué representaba. Boltzmann propone una función H que da la entropía correcta cuando no hay interacción entre las partículas del sistema; su idea detrás de esta función es simplemente tomar la densidad en el espacio de fase de una partícula y multiplicarla por $N$, este es el número de componentes en el cuerpo macroscópico. La función de Boltzmann es:

\begin{equation}
H_{B}=N \int w_{1} \log w_{1} d \tau_{1}.
\end{equation}
donde $w_{1}$ es la densidad de probabilidad de una partícula y $d \tau_{1} \equiv d^{3}x_{1}d^{3}p_{1}$.
Pero al introducir la interacción la función de Boltzmann no da la entropía correcta, es menor que la experimental. Gibbs por su parte propone una función H que sí da la entropía correcta para cualquier caso pero las ideas que tiene esta función de Gibbs no son tan directas. Gibbs le da sentido a su función hablando de las copias del microestado posible, con esto da una densidad de estados en el espacio de fase y luego define sus función.
\begin{equation}
H_{G}= \int W_{N}  \log W_{N} d \tau
\end{equation}
donde $W_{N}$ es la densidad de probabilidad en el espacio de fase y $d\tau \equiv d^{3}x_{1}...d^{3}p_{N}$. Aunque la perspectiva de Gibbs no sea tan directa da los resultados correctos. Esta variedad de perspectivas hace que no se tenga un marco conceptual único en la mecánica estadística.
\\
Otro problema que ocurre es que el principio en el que se basa la mecánica estadística es el de probabilidades iguales. Esto es que al no conocer el estado exacto en el que se encuentra el sistema se supone por la ignorancia subjetiva que todos los estados deben tener la misma probabilidad de ocurrir. Dejar toda una teoría física en el cimiento de una probabilidad subjetiva no es deseable.
\\
Además de estos problemas existe el conflicto sobre los promedios de ensamble. Con las ideas de Gibbs se puede encontrar promedios pero como la función de densidad es de las copias de los microestados, ¿ qué significa un promedio de ensamble?. Se toma el camino ergódico donde se supone que el promedio de ensamble es igual al promedio temporal, este supuesto se fundamenta en que el estado va a pasar por todos lo microestados posbiles en el espacio de fase. El conflicto con eso es que hay sistemas que no pasan por todos los microestados posibles pero aún así se puede manejar estos sistemas; el tiempo necesario para que el estado pase por todos los microestados es bastante largo. Esto problemas son los que dejan al marco teórico de la mecánica estadística como algo controversial. 

\end{comment}


