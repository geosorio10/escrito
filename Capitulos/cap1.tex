\chapter{El entrelazamiento y la mecánica estadística}

La idea principal que se quiere mostrar es cómo se puede reemplazar el postulado de probabilidades iguales por un principio canónico general basado en el entrelazamiento cuántico, basado en el entrelazamiento del sistema y el ambiente.

La mecánica clásica nos habla de un sistema físico definido que para todos los tiempos está especificado. Este sistema evolucionan de manera determinista. Pero lo que sorprende al tratar sistemas termodinámicos es que aunque se hable de un sistema clásico este puede mostrar propiedades que dependan de promedios estadísticos. La conexión que hay entre el determinismo y estas probabilidades es una discusión que lleva desde los inicios de la termodinámica. Aunque el planteamiento que se mostrará no necesita de los métodos típicos de la mecánica estadística como por ejemplo: aleatoriedad subjetiva, promedio sobre el ensamble o promedio temporal; Esto le da una fuerza a estas ideas ya que no debe entrar en problemas de ergodicidad. 

Este nuevo enfoque se tiene un universo que está compuesto por el sistema y el ambiente. Dando como condición que el ambiente sea los suficientemente grande. Este universo está descrito por un estado cuántico puro ( se conoce el estado de manera exacta) que obedece una restricción global, que el sistema alcance el equilibrio térmico por medio de la interacción mutua (termalización) es un producto del entrelazamiento del sistema y el ambiente (Popescu et al,2006). Más adelante se le dará una definición más rigurosa que ayudará a dar cotas para la expresión "ambiente suficientemente grande". Con este enfoque se quiere formular un principio canónico general  el sistema estará termalizado para casi todos los estados puros del universo, dando límites cuantitativos. La restricción que se impone no es una específica esto generaliza los resultados tradicionales dados en la literatura donde se toma por restricción la energía (cualquier libro, año).
Estos resultados no miran la evolución del sistema pero debido a que la mayoría de los estados del universo están termalizados se preve que no importa el estado inicial cualquier evolución llevará al estado a uno en el equilibrio.

(
comentario sobre esto.
)

Ya poniendo las ideas más explícitas se supone tener un sistema cuántio aislado y grande que se llamará el universo, que se descompone en dos partes el sistema $S$ y el ambiente $E$. La dimensión del ambiente es mucho maś grande que la del sistema $S$. Además se le impone una restricción global al universo llamada $R$. Desde la mecánica cuántica estas condiciones se pueden poner como restricciones en el espacio de Hilbert, restricción de los estados posibles:

\begin{equation}
\mathcal{H_{R}}\subseteq \mathcal{H_{S}}\otimes \mathcal{H_{E}},
\end{equation}

Donde $\mathcal{H_{S}}$ y $\mathcal{H_{E}}$ son los espacios de Hilbert del sistema y el ambiente con dimension $d_{\mathcal{S}}$  y $d_{\mathcal{E}}$ respectivamente. Es bueno recalcar que $R$ es una restricción arbitraria generalmente se toma como la energía del universo.
Ahora se define el estado equiprobable del universo bajo $R$ como

\begin{equation}
\mathcal{E_{R}} = \frac{1}{d_{R}} \mathbbm{1_{\mathcal{R}}},
\end{equation}

Donde $\mathbbm{1_{\mathcal{R}}}$ es el operador identidad (proyección) sobre el espacio de Hilbert  $\mathcal{H_{R}}$  que tiene dimensión $d_{\mathcal{R}}$. Esto se relaciona con el principio de probabilidades iguales porque este es el estado máximamente mezclado en $\mathcal{H_{R}}$ por ser así todos los estados bajo la restricción de R tienen la misma probabilidad de salir.

Definimos $\Omega_{\mathcal{S}}$ como el estado canónico que está restringido por $R$ cuando el universo se encuentra en el estado $\mathcal{E_{R}}$.Esto significa que si se hace una traza parcial al universo sobre el ambiente da como resultado el estado canónico:
\begin{equation}
 \Omega_{\mathcal{S}} =Tr_{\mathcal{E}} \mathcal{E_{R}}.
\end{equation}

Aquí se hace un supuesto importante y es que el universo está en un estado puro $\ket{\phi}$ y no en un estado mixto $\mathcal{E_{R}}$, esto quiere decir que se conoce todo lo que es permitido por la mecánica cuántica del universo si estuviese en un estado mixto significaría que nosotros no tenemos toda la información que se pudiese tener. Ahora lo que se quiere ver es que aun que el estado del universo sea puro el estado reducido del sistema,

\begin{equation}
\rho_{S}=Tr_{E}\ket{\phi}\bra{\phi} 
\end{equation}

se acerca al estado canónico para la gran mayoría de los casos 

\begin{equation}
\rho_{S} \approx \Omega_{S}.
\end{equation}

O sea que para casi todos los estados puros $\ket{\phi} \in \mathcal{H_{R}}$ del universo el sistema se comporta como si el universo estuviese en el estado mixto equiprobable $\mathcal{E_{R}}$. Siendo este el principio general canónico.

Clarificando la idea, el estado canónico $\Omega_{S}$ del sistema es el estado del sistema cuando el universo se encuentra en el estado equiprobable $\mathcal{E_{R}}$. Se puede interpretar el principio general canónico como un principio que estipula que las probabilidades iguales del sistema son aparentes. Porque para casi cualquier estado del universo que sea puro, un subsistema de este universo que cumpla con ser lo suficientemente pequeño se comporta como si el universo estuviese en el estado equiprobable $\mathcal{E_{R}}$. Cabe recordar que aún no se ha especificado la restricción $R$ entonces todo este análisis es general, la restricción no necesariamente debe ser la energía u otras cantidades que se conserven. Esto hace que $\Omega_{S}$ no  deba ser obligatoriamente es el estado canónico usual, puede ser el gran canónico o cualquier otro que sea acorde con la restricción. Este principio puede ser de utilidad cuando la interacción entre el ambiente y el sistema no es débil como siempre se toma.

\section{formulación matemática}

Ahora se procederá a especificar lo dicho anteriormente en un contexto matemático. Lo primordial es decir cuál será la distancia que usaremos para darle un sentido de cercanía a los estados $\rho_{S}$ y $\Omega_{S}$. La distancia a usar es una bastante conocida en la teoría cuántica de la información, la distancia de traza. Esta se define como:
\begin{equation}
D(\rho_{S}, \Omega_{S})= \frac{1}{2} Tr |\rho_{S} -\Omega_{S}|=\frac{1}{2}Tr\sqrt{(\rho_{S} -\Omega_{S})^{\dag}(\rho_{S} -\Omega_{S})}.
\end{equation}

Esta medida cuantifica que tan difícil es diferenciar $\rho_{S}$ y $\Omega_{S}$ por mediciones cuánticas. Un poco de notación $\langle . \rangle$ es el promedio sobre todos los estados $\ket{\phi} \in \mathcal{H_{R}}$ de acuerdo a la medida estandar(unitariamente invariante). Esta medida se usa para hallar volúmenes de conjuntos de estados. Luego por esto se sabe que $\Omega_{S} = \langle \rho_{S} \rangle $.

Entonces el teorema central de este capítulo es:\\

\textbf{Teorema 1:}Para un estado escogido de manera aleatoria $\ket{\phi} \in \mathcal{H_{R}} \subseteq \mathcal{H_{S}} \otimes \mathcal{H_{E}} $ y un $\epsilon > 0$ arbitrario, la distancia entre la matriz densidad reducida del sistema $\rho_{S}=Tr(\ket{\phi} \bra{\phi})$  y el estado canónico $\Omega_{S}=Tr \mathcal{E_{R}}$ esta dado probabilísticamente por 

\begin{equation}
Prob[  \norm{\rho_{S} -\Omega_{S}}_{1} \geq \eta ] \leq \eta',
\end{equation}

Donde 

\begin{equation}
\norm{\rho_{S}-\Omega_{S}}_{1}=2 D(\rho_{S}, \Omega_{S})
\end{equation}

\begin{equation}
\eta= \epsilon + \sqrt{ \frac{d_{S}}{d_{E}^{eff}} },
\end{equation}

\begin{equation}
\eta'=2exp(-Cd_{R}\epsilon^{2}).
\end{equation}

y las constantes son: $ C=(18\pi^{3})^{-1}, d_{R} = dim \mathcal{H_{R}}, d_{S} = dim \mathcal{H_{S}} $. $d_{E}^{eff}$ es la medida efectiva del tamaño del ambiente,
\begin{equation}
d_{E}^{eff}= \frac{1}{Tr \Omega_{E}^{2}} \ge \frac{d_{R}}{d_{S}}.
\end{equation}

Donde $\Omega_{E}= Tr_{S} \mathcal{E_{R}}$. Ambas cantidades $\eta $ y $\eta'$ serán pequeñas esto implica que el estado estará cercano al estado canónico con alta probabilidad cuando la dimensión efectiva del ambiente sea mucho más grande que la del sistema ( es decir $d_{E}^{eff} >> d_{S}$) y  $d_{R}\epsilon^2>>1>>\epsilon$. Esta última condición se puede asegurar cuando el espacio acesible total es grande ( es decir $d_{R}>>1$), escogiendo $\epsilon=d_{r}^{-\frac{1}{3}}$.\\

Ya con esto se le da un significado cuantitativo de lo que se ha ido explicando hasta ahora. Además también se dan unas cotas explícitas a cuando se habla de la "mayoría" de los estados. Se tiene una cota exponencialmente pequeña  sobre la probabilidad de encontrar un estado lejano del canónico. (Dar comentario sobre lo intuitivo de las cotas)\\
\section{Lema de Levy}
El teorema anterior tiene como base principal el lema de Levy (referencias). Este dice que al seleccionar un punto $\phi$ aleatoriamente de una hiperesfera de dimensión alta y que $f(\phi)$ no cambie muy rápido, entonces $f(\phi) \approx \langle f \rangle $ con alta probabilidad, Más exactamente:\\

\textbf{Lema de Levy:} Dada una función $f: \mathbb{S}^d \to \mathbb{R} $ definida en la hiperesfera $\mathbb{S}^d$, y un punto $\phi \in \mathbb{S}^d $ es escogido de manera aleatoria uniforme, 
\begin{equation}
Prob[|f(\phi)- \langle f \rangle| \ge \epsilon ] \le 2exp(-\frac{2C(d+1)\epsilon^2}{\eta^2})
\end{equation}
donde $\eta$ es la constante de Lipschitz de $f$, dado por $\eta= \sup|\nabla f|$ y $C=(18\pi^3)^-1$.\\

Gracias a la normalización, los estados puros en $\mathcal{H_{R}}$ se pueden representar como puntos sobre la superficie de una hiperesfera de dimensión $2d_{R}-1$, o sea  $\mathbb{S}^{2d_{R}-1}$. Luego e puede aplicar este lema a estado cuánticos $\phi$ aleatoriamente seleccionados. Para los estados seleccionados aleatoriamente $\phi \in \mathcal{H_{R}}$, se desea mostrar que $\norm{\rho_{S}- \Omega_{S}}_{1} \approx 0$ con alta probabilidad.\\

\section{Demostración del teorema 1}
Ya con el teorema de Levy como herramienta se usará este para poder demostrar el teorema 1. Para usar el teorema de Levy para el problema actual se dice que $f(\phi)= \norm{\rho_{S} - \Omega_{S}}_{1}$. Antes de seguir con la demostración se quiere dar la idea de qué habla la constante de Lipschitz que fue nombrada anteriormente. Para esto se debe ver qué significa que una función sea Lipschitz continua. \\
La definición general de continuidad es: Sea $f: I \to \mathbb{R}$ donde $I$ puede ser un intervalo abierto $(a,b)$ o uno cerrado $[a,b]$, además $C \in I$. Se dice que $f$ es continua en $C \longleftrightarrow \forall \epsilon >0 $ $\exists$ $ \delta >0 $ tal que  $ |x-c|\longrightarrow |f(x)-f(c)|< \epsilon $. Esta es la definición que en general se maneja pero hay sutilezas en esta definición que no siempre son mostradas como por ejemplo que $\delta$ depende de donde se ponga el  punto $C$, esto se ve claramente en la siguiente función: $f: (0,1) \to \mathbb{R}$, $f(x)=\frac{1}{x} $ al $C$ estar más lejos del $0$ permite un $\delta$ más grande pero al acercarse al $0$ el $\delta$ debe ser más pequeño.
La continuidad de Lipschitz permite que se defina un $\delta$ constante sin importar donde se encuentre el $C$. Se es Lipschitz continuo si 

\begin{equation}
|f(x) -f(y)| \le \eta |x-y|,
\end{equation}

 con $\eta$ constante esto permite poner $|f(x)-f(y)| < \epsilon \to $ $\delta < \frac{\epsilon}{\eta} $. Ahora si f es derivable y $\nabla f$ es acotado, $x$ y $y$ dados existe $\xi$ entre $x$,$y$ tal que:
 
\begin{equation}
f(x)-f(y)= \nabla f(\xi) (x-y)
\end{equation}

\begin{equation}
\to |f(x)- f(y)| \le |\nabla f(\xi)| |x-y|
\end{equation}

\begin{equation}
\to |f(x)- f(y)| \le \sup|\nabla f(\xi)| |x-y|
\end{equation}
 
Como $\nabla f$ es acotado se tiene que $\sup |\nabla f(\xi)|=\eta$.\\

Dado esta introducción a la continuidad de Lipschitz va encontrará una cota para la constante $\eta$ de la función $f(\phi)=\norm{\rho_{S}-\Omega_{S}}_{1}$. Entonces para poder lograr esto se procederá de la siguiente forma se definen dos estados reducidos $\rho_{1}= Tr_{E} (\ket{\phi_{1}} \bra{\phi_{1}})$ y $\rho_{2}=Tr_{E}(\ket{\phi_{2}} \bra{\phi_{2}})$, Ahora se tiene

\begin{equation}
|f(\phi_{1})-f(\phi_{2})|^2= |\norm{\rho_{1}-\Omega}_{1} - \norm{\rho_{2}-\Omega}_{1}|^2.
\end{equation}

como $\norm{M}_{1}$ es una distancia (esto es $d(\rho_{1},\Omega)= \norm{\rho_{1}-\Omega}_{1}$) es cierto para un espacio métrico

\begin{equation}
|d(x,z)-d(y,z)| \le d(x,y)
\end{equation}

Por lo tanto 

\begin{equation}
|\norm{\rho_{1}-\Omega}_{1} - \norm{\rho_{2}-\Omega}_{1}|^2 \le \norm{\rho_{1}- \rho_{2} }_{1}^2=\norm{Tr_{E}(\ket{\phi_{1}} \bra{\phi_{1}}-\ket{\phi_{2}} \bra{\phi_{2}})}_{1}^{2}.
\end{equation}

Como hay una cota a la norma de una traza parcial,

\begin{equation}
\norm{Tr_{\mathcal{B}}(M)}_{p} \le [dim(\mathcal{H_{B}})]^{\frac{p-1}{p}} \norm{M}_{p}.
\end{equation}

entones por la cota anterior

\begin{equation}
\norm{Tr_{E}(\ket{\phi_{1}} \bra{\phi_{1}}-\ket{\phi_{2}} \bra{\phi_{2}})}_{1} \le \norm{\ket{\phi_{1}} \bra{\phi_{1}}-\ket{\phi_{2}} \bra{\phi_{2}}}_{1}.
\end{equation}

Se tiene que :
\begin{equation}
\norm{\rho_{1}- \rho_{2} }_{1}^2 \le \norm{\ket{\phi_{1}} \bra{\phi_{1}}-\ket{\phi_{2}} \bra{\phi_{2}}}_{1}^{2}
\end{equation}

Usando la hermiticidad de $\rho$ y el teorema espectral se puede descomponer $\rho= UDU^{\dag}$ donde $U$ es un operador unitario y $D$ es diagonal. Además por las propiedades de la traza $Tr(\sqrt{UD^2U^{\dag}})= Tr(U\sqrt{D^2}U^{\dag})= Tr(\sqrt{D^{2}})$ se llega a
\begin{equation}
\norm{\ket{\phi_{1}} \bra{\phi_{1}}-\ket{\phi_{2}} \bra{\phi_{2}}}_{1}^{2} = 4(1-|\bra{\phi_{1}} \ket{\phi_{2}}|^{2})
\end{equation}
también
\begin{equation}
 4(1-|\bra{\phi_{1}} \ket{\phi_{2}}|^{2}) \le 4|\ket{\phi}-\ket{\phi}|^2.
\end{equation}

Uniendo todo los pasos anteriores
\begin{equation}
|f(\phi_{1})-f(\phi_{2})|^{2} \le  4|\ket{\phi}-\ket{\phi}|^2
\end{equation}

o sea 
\begin{equation}
|f(\phi_{1})-f(\phi_{2})| \le  2|\ket{\phi}-\ket{\phi}|,
\end{equation}

Con esto se muestra que $\eta \le 2$.\\

Habiendo dado una cota para $\eta$ ahora se usará el lema de Levy para la función $f(\phi)=\norm{\rho_{S} -\Omega_{S}}_{1}$ recordando que se reemplazará $d$ en el lema por $d=2d_{R}-1$. sustituyendo en el teorema se tiene:

\begin{equation}
 2exp(-\frac{2C(d+1)\epsilon^2}{\eta^2})= 2exp(-\frac{4Cd_{R}\epsilon^2}{\eta^2})
\end{equation}

como $\eta \le 2$ 

\begin{equation}
2exp(-Cd_{R}\epsilon^2) \ge 2exp(-\frac{4Cd_{R}\epsilon^2}{\eta^2}) \ge Prob[|f(\phi)- \langle f \rangle| \ge \epsilon ]
\end{equation}

mirando más atentamente  $|f(\phi)- \langle f \rangle| \ge \epsilon $, por ser  una distancia $\norm{\rho_{S} -\Omega_{S}}_{1} \ge 0$ entonces


\begin{equation}
\norm{\rho_{S} -\Omega_{S}}_{1}- \langle \norm{\rho_{S} -\Omega_{S}}_{1} \rangle \ge \epsilon.
\end{equation}

Nombrando a $\mu = \epsilon + \langle \norm{\rho_{S} -\Omega_{S}}_{1} \rangle $ y $\mu'=2 exp(-Cd_{R}\epsilon^2)$ esto permite organizar lo anterior así:

\begin{equation}
Prob[\norm{\rho_{S} -\Omega_{S}}_{1} \ge \mu] \le \mu'.
\end{equation}
Ahora para poder obtener el resultado se debe acotar $\mu$ .Se asegura que $\epsilon$ y $\mu'$ son cantidades pequeñas al escoger $\epsilon=d_{R}^{-1/3}$ porque $d_{R}>>1$. Se quiere ahora ponerle un cota a $\langle \norm{\rho_{S} -\Omega_{S}}_{1} \rangle$ para esto primero se acotará dando los límites por las trazas y luego se calcularán estas para poder dejar la cota en terminos de las dimensiones del sistema. Primero se procederá a encontrar la relación entre $\norm{\rho_{S} -\Omega_{S}}_{1}$ y $\norm{\rho_{S} -\Omega_{S}}_{2}$ esto se hará para tener una facilidad de manejo mayor ya que la segunda norma(Hilbert-Schmidt) permite un mejor tratamiento.\\
La relación entre estas dos normas se puede ver desde el manejo de amtrices. Sea $M$ una matriz $nxn$ se sabe que si $M$ tiene $\lambda_{i}$ valores propios entonces:
\begin{equation}
Tr M= \sum_{i} \lambda_{i} 
\end{equation}
con esto se puede escribir de manera explícita la norma de traza 
\begin{equation}
\norm{M}_{1}^2=(Tr|M|)^{2}=n^{2} (\frac{1}{n}\sum_{i} |\lambda_{i}|)^{2}.
\end{equation}
Como la función $x^{2}$ es convexa se puede usar la desigualdad de Jensen que	



(poner luego de las demostraconesCuando  el ambiente es mucho más grande que el sistema $\mu$ y $\mu'$ serán pequeñas ($d_{E}^{eff}>>d_{S}$) implicando $\norm{\rho_{S} -\Omega_{S}}_{1} \approx 0$ con alta probabilidad.)












